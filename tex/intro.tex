Dabiskā valodas apstrāde (NLP -- \textit{natural language processing}) ir starpdisciplināra datorlingvistikas un mākslīgā intelekta nozare, kas strādā pie tā, lai datori varētu saprast cilvēka dabiskās valodas ievadi. Dabiskās valodas pēc būtības ir sarežģītas, un daudzi NLP uzdevumi ir slikti piemēroti matemātiski precīziem algoritmiskajiem risinājumiem. Palielinoties korpusu -- liela apjoma rakstītas vai runātas dabiskās valodas kolekciju -- pieejamībai, NLP uzdevumi arvien biežāk un efektīvāk tiek risināti ar mašīnmācīšanās modeļiem \cite{nlp2018}.


Arvien lielāku daļu tirgus pārņem pakalpojumu industrija,
%(\% ES)
un pakalpojumi arvien biežāk tiek piedāvāti starptautiski. Tam ir nepieciešams lietotāju dzimtās valodas atbalsts gan valstu valodu regulējumu, gan tirgus nišas ieņemšanas un tirgus konkurences dēļ.
%(\% ES iedzīvotāju svarīgi saņemt pakalpojumu savā dzimtajā valodā). 

Dabiskās valodas apstrādei ir liels biznesa potenciāls, jo tas ļauj uzņēmumiem palielināt peļņu samazinot izdevumus, no kuriem lielākais parasti ir darbs. %\% pakalpojumu nozares uzņēmumu izdevumi ir darbs. Var minēt tech layoffs, specifically, Meta’s Year of Efficiency % original sentence: Uzņēmumiem tas ir izdevīgi, jo ļauj samazināt personālizdevumus.
% (\% pakalpojumu nozares uzņēmuma izdevumu)
Tas savukārt samazina barjeru iekļūšanai un dalībai starptautiskā tirgū, kas nozīmē lielāku piedāvāto pakalpojumu daudzveidību un konkurenci, tātad zemākas izmaksas patērētājam. Lietotājiem, kuru dzimto valodu pārvalda mazs cilvēku skaits kā tas ir, piemēram, latviešu valodā, kļūst pieejami pakalpojumi, kuru tulkojumus būtu ekonomiski nerentabli nodrošināt ar algotu profesionālu personālu.

Darbā apskatītā metode nodrošina automatizāciju divos veidos: 
\begin{itemize}
	\item daudzvalodīgs modelis aizvieto profesionālu tulkotāju;
	\item virtuālais asistents aizvieto klientu apkalpošanas speciālistu.
\end{itemize}

Darbs ir sadalīts teorētiskajā un praktiskajā daļā. Teorētiskajā daļā ir īsi aprakstīti mūsdienu modeļi un pieejas. Praktiskajā daļā ir veikti eksperimenti ar mērķi pielietot daudzvalodīgus modeļus un salīdzināt tos ar esošiem risinājumiem.

Pētījuma jautājums: Kādas ir efektīvākās metodes un daudzvalodīgi jēdzientelpu modeļi daudzvalodu nodomu noteikšanai?