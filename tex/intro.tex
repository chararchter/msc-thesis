Arvien lielāku daļu tirgus pārņem pakalpojumu industrija,
%(\% ES)
un pakalpojumi arvien biežāk tiek piedāvāti starptautiski. Tam ir nepieciešams lietotāju dzimtās valodas atbalsts gan valstu valodu regulējumu, gan tirgus nišas ieņemšanas un tirgus konkurences dēļ.
%(\% ES iedzīvotāju svarīgi saņemt pakalpojumu savā dzimtajā valodā). 

Uzņēmumiem tas ir izdevīgi, jo ļauj samazināt personālizdevumus.
% (\% pakalpojumu nozares uzņēmuma izdevumu)
Tas savukārt samazina barjeru iekļūšanai un dalībai starptautiskā tirgū, kas nozīmē lielāku konkurenci un piedāvāto pakalpojumu daudzveidību. Lietotājiem, kuru dzimto valodu pārvalda mazs cilvēku skaits kā tas ir, piemēram, latviešu valodā, kļūst pieejami pakalpojumi, kuru tulkojumus būtu ekonomiski nerentabli nodrošināt ar algotu profesionālu personālu.

Darbā apskatītā metode nodrošina automatizāciju divos veidos: 
\begin{itemize}
	\item daudzvalodīgs modelis aizvieto profesionālu tulkotāju;
	\item virtuālais asistents aizvieto klientu apkalpošanas speciālistu.
\end{itemize}

Darbs ir sadalīts teorētiskajā un praktiskajā daļā. Teoretiskajā daļā ir īsi aprakstīti mūsdienu modeļi un pieejas. Praktiskajā daļā ir veikti eksperimenti ar mērķi pielietot daudzvalodīgus modeļus un salīdzināt tos ar esošiem risinājumiem.

Pētījuma jautājums: Kādas ir efektīvākās metodes un daudzvalodīgi jēdzientelpu modeļi daudzvalodu nodomu noteikšanai?