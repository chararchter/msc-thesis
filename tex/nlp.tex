Dabiskā valodas apstrāde (NLP – natural language processing) ir starpdisciplināra datorlingvistikas un mākslīgā intelekta nozare, kas strādā pie tā, lai datori varētu saprast cilvēka dabiskās valodas ievadi. Dabiskās valodas pēc būtības ir sarežģītas, un daudzi NLP uzdevumi ir slikti piemēroti matemātiski precīziem algoritmiskajiem risinājumiem. Palielinoties korpusu (liela apjoma rakstītas vai runātas dabiskās valodas kolekcija) pieejamībai, NLP uzdevumi arvien biežāk un efektīvāk tiek risināti ar mašīnmācīšanās modeļiem \cite{nlp2018}. Dabiskās valodas apstrādei ir liels biznesa potenciāls, jo tas ļauj uzņēmumiem palielināt peļņu samazinot izdevumus, no kuriem lielākais parasti ir darbs.


Viens no svarīgākajiem korpusiem tieši nodomu noteikšanā ir aviokompāniju ceļojumu informācijas sistēmu (ATIS - Airline Travel Information Systems) datu kopa. Tā ir audioierakstu un manuālu transkriptu datu kopa, kas sastāv no cilvēku sarunām ar automatizētām aviolīniju ceļojumu informācijas sistēmām. ATIS datu kopa nodrošina lielu ziņojumu un ar tiem saistīto nodomu skaitu, ko plaši izmanto kā novērtējuma (benchmark) datu kopu klasifikatoru apmācībai nodomu noteikšanā \cite{atis1990}.
