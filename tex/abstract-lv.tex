Daudzvalodīga lietotāja nodomu noteikšana ir nozīmīga virtuālo asistentu darbībā, un klientu apkalpošanas automatizācija kļūst arvien izdevīgāka un aktuālāka. Viens veids noteikt nodomu ir attēlojot ievades teksta virknes daudzdimensionālā vektoru telpā jeb jēdzientelpā, kuru izmanto nodomu klasifikācijas modeļi, lai piegādātu lietotājiem tiem nepieciešamo informāciju. Darbā tiks apmācīti dažādi mašīnmācīšanās modeļi un salīdzinātas dažādas pieejas: ievades teksta attēlojums uz daudzvalodīgu tekstu korpusu apmācītas jēdzientelpas un ievades teksta mašīntulkošana uz angļu valodu un attēlojums uz angļu valodas korpusa apmācītas jēdzientelpas.

\keywords{daudzvalodīgas jēdzientelpas, nodomu noteikšana}
