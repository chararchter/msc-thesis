Daudzvalodīga lietotāja nodomu noteikšana ir nozīmīga virtuālo asistentu darbībā, un klientu apkalpošanas automatizācija kļūst arvien aktuālāka. Ievades teksta virknes tiek attēlotas daudzdimensionālā vektoru telpā jeb jēdzientelpā, kuru izmanto nodomu klasifikācijas modeļi, lai piegādātu lietotājiem tiem nepieciešamo informāciju. Darbā tiks apmācīti dažādi mašīnmācīšanās modeļi un salīdzinātas dažādas pieejas, piemēram, ievades teksta attēlojums uz daudzvalodīgu tekstu korpusu apmācītas jēdzientelpas vai ievades teksta mašīntulkošana uz angļu valodu pirms jēdzientelpas izveides.

\keywords{daudzvalodīgas jēdzientelpas, nodomu noteikšana}