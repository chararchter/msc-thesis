Daudzvalodīga lietotāja nodomu noteikšana ir būtiska virtuālo asistentu darbībā, un klientu apkalpošanas automatizācija kļūst arvien izdevīgāka un aktuālāka. 

Lietotāju nodomi tiek noteikti vispirms attēlojot lietotāja ievadīto tekstu daudzdimensionālā vektoru telpā jeb jēdzientelpā. Tad mašīnmācīšanās modelis klasificē vektorā izteikto nodomu, lai piegādātu lietotājiem nepieciešamo informāciju.

Darbā tiek izmantots anotēts nodomu noteikšanas korpuss, kas satur lietotāja ievadu un nodomu pārus angļu, latviešu, krievu, igauņu un lietuviešu valodās. Pētījumā tiek salīdzināta nodomu noteikšanas precizitāte divām daudzvalodu jēdzientelpām (ģenerētas ar mBERT un XLM-RoBERTa modeļiem) un divu veidu ievaddatu valodām (oriģinālvalodā un mašīntulkojumā uz angļu valodu), testējot nodomu klasifikācijas modeli, kas apmācīts:
\begin{itemize}
    \item uz tās pašas valodas korpusa;
    \item uz visu piecu valodu korpusa;
    \item tikai uz angļu valodas korpusa.
\end{itemize}

Rezultāti liecina par to, ka daudzvalodīgas jēdzientelpas un apmācības uz daudzvalodu korpusiem var uzlabot nodomu noteikšanas precizitāti, bet atkarībā no valodas var būt atšķirīgi rezultāti.

\keywords{daudzvalodīgas jēdzientelpas, nodomu noteikšana, mBERT, XLM-RoBERTa}