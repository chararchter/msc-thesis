Nodoms ir mērķis, kas lietotājam ir padomā, rakstot jautājumu. Nodomu noteikšana ir lietotāja ievades teksta klasifikācija tam piešķirot visvarbūtīgāko nodomu no iepriekš definētu nodomu kopas \cite{fasttext2019}. Piemēram, klasificējot lietotāja nodomu kā vilciena atiešanas laiks, čatbots var sniegt lietotājam nepieciešamo atbildi no vilcienu grafika (\ref{tab:input-intent} tabula).


\begin{table}[htbp]
	\centering
	\caption{Lietotāja ievada un nodoma piemērs}
	\begin{tabular}{ll}\toprule
		Ievads & Cikos ir nākošais vilciens no Rīgas uz Siguldu? \\
		Nodoms & vilciena atiešanas laiks \\\bottomrule
	\end{tabular}%
	\label{tab:input-intent}%
\end{table}

% kā dara nodomu noteikšanu?
Pirms mašīnmācīšanās nodomi tika noteikti ar šabloniem (\textit{pattern-based recognition}), bet izveidot un uzturēt lielu skaitu šablonu ir darbietilpīgi. Advancētāka pieeja nodomu noteikšanai ir apmācīt neironu tīklu klasifikatoru uz anotētas datu kopas -- lietotāju ievades tekstiem un atbilstošajiem klientu apkalpošanas speciālista identificētajiem lietotāja nodomiem. Ierobežotās apmācību kopas dēļ dialogsistēmas/virtuālie asistenti var atbildēt uz ierobežotu jautājumu klāstu, piemēram, aptverot bieži uzdotos jautājumus (FAQ -- \textit{Frequently Asked Questions}) \cite{fasttext2019}.

Lai arī jaunāko valodas modeļu, piemēram, GPT-3, izvades teksti lietotājam rada iespaidu par tekošu valodu, pastāv neparastās ielejas (\textit{uncanny valley}) efekts, kurā novērotā plūstošā atbildes valoda rada ekspektācijas, kuras virtuālie asistenti nevar attaisnot un izraisa neapmierinātību \cite{paikens2020}. Tāpēc klienta nodoma noteikšana ir svarīga, lai nodrošinātu patīkamu lietotāja pieredzi.

Jāpiebilst, ka labuma gūšanai no nodomu noteikšanas automatizācijas nav nepieciešams pārklāt 100\% lietotāju pieprasījumu. Veiksmīgas izmantošanas piemērs telekomunikāciju industrijā validācijā izmantoja 1732 klientu pieprasījumu datu kopu anotētu ar attiecīgajiem nolūkiem. Šajā gadījumā divi visbiežākie nodomi ir rēķina atlikšana (356 pieprasījumi; 21\% datu kopas) un nokavēta rēķina maksājuma apstiprināšana (207 pieprasījumi; 12\% datu kopas). Trīs mēnešus ilgā eksperimentālā pētījuma tika apstrādāti 14000 lietotāju pieprasījumi. Sākotnējos testos nodomu noteikšana un izvēlētā atbildes veidne bija precīza 90\% gadījumu, eksperimenta gaitā iegūtie dati ļāva uzlabot nodomu noteikšanu par 2\%, tātad klientu apkalpošanas speciālistiem bija jāveic izmaiņas tikai 8\% pieprasījumu rēķinu kategorijā \cite{paikens2020}.

Tipiski soļi nodomu noteikšanas pielietojumam uzņēmējdarbībā:
\begin{enumerate}
	\item Atrast visbiežākos pieprasījumu tipus;
	\item Sagatavot atbildes veidni (\textit{template});
	\item Nodomu noteikšanas sistēma identificē, vai lietotāja pieprasījums pieder iepriekš definētajaiem tipiem un izdod potenciālo atbildi;
	\item Klientu apkalpošanas speciālists izvērtē un koriģē atbildi pirms nosūtīšanas;
	\item Automātiski uzlabot nodomu noteikšanas sistēmu, balstoties uz speciālista veiktajām korekcijām \cite{paikens2020}.
\end{enumerate}
