Nodoms ir mērķis, kas lietotājam ir padomā, rakstot jautājumu. Nodomu noteikšana ir lietotāja ievades teksta klasifikācija tam piešķirot visvarbūtīgāko nodomu no iepriekš definētu nodomu kopas \cite{fasttext2019}. Piemēram, klasificējot lietotāja nodomu kā vilciena atiešanas laiks, čatbots var sniegt lietotājam nepieciešamo atbildi no vilcienu grafika (\ref{tab:input-intent} tabula). Dabiskā valodā ir vairāki veidi kā izteikt vienu un to pašu nodomu (\ref{tab:input-intent-ambiguous} tabula).


\begin{table}[htbp]
	\centering
	\caption{Lietotāja ievada un nodoma piemērs}
	\begin{tabular}{ll}\toprule
		Ievads & Cikos ir nākošais vilciens no Rīgas uz Siguldu? \\
		Nodoms & vilciena atiešanas laiks \\\bottomrule
	\end{tabular}%
	\label{tab:input-intent}%
\end{table}


\begin{table}[htbp]
	\centering
	\caption{Lietotāja dažādi ievadi ar vienu nodomu \cite{snips-docs}}
	\begin{tabular}{ll}\toprule
		Nodoms & Ievads \\
		switchLightOn & Ieslēdz gaismu \\
		switchLightOn & Istabā ir pārāk tumši, vai vari to izlabot? \\\bottomrule
	\end{tabular}%
	\label{tab:input-intent-ambiguous}%
\end{table}

% kā dara nodomu noteikšanu?
Pirms mašīnmācīšanās nodomi tika noteikti ar šabloniem (\textit{pattern-based recognition}), bet izveidot un uzturēt lielu skaitu šablonu ir darbietilpīgi. Advancētāka pieeja nodomu noteikšanai ir apmācīt neironu tīklu klasifikatoru uz anotētas datu kopas -- lietotāju ievades tekstiem un atbilstošajiem klientu apkalpošanas speciālista identificētajiem lietotāja nodomiem. Ierobežotās apmācību kopas dēļ dialogsistēmas/virtuālie asistenti var atbildēt uz ierobežotu jautājumu klāstu, piemēram, aptverot bieži uzdotos jautājumus (FAQ -- \textit{Frequently Asked Questions}) \cite{fasttext2019}.

Lai arī jaunāko valodas modeļu, piemēram, GPT-3, izvades teksti lietotājam rada iespaidu par tekošu valodu, pastāv neparastās ielejas (\textit{uncanny valley}) efekts, kurā novērotā plūstošā atbildes valoda rada ekspektācijas, kuras virtuālie asistenti nevar attaisnot un izraisa neapmierinātību \cite{paikens2020}. Tāpēc klienta nodoma noteikšana ir svarīga, lai nodrošinātu patīkamu lietotāja pieredzi.

Jāpiebilst, ka labuma gūšanai no nodomu noteikšanas automatizācijas nav nepieciešams pārklāt 100\% lietotāju pieprasījumu. Veiksmīgas izmantošanas piemērs telekomunikāciju industrijā validācijā izmantoja 1732 klientu pieprasījumu datu kopu anotētu ar attiecīgajiem nolūkiem. Šajā gadījumā divi visbiežākie nodomi ir rēķina atlikšana (356 pieprasījumi; 21\% datu kopas) un nokavēta rēķina maksājuma apstiprināšana (207 pieprasījumi; 12\% datu kopas). Trīs mēnešus ilgā eksperimentālā pētījuma tika apstrādāti 14000 lietotāju pieprasījumi. Sākotnējos testos nodomu noteikšana un izvēlētā atbildes veidne bija precīza 90\% gadījumu, eksperimenta gaitā iegūtie dati ļāva uzlabot nodomu noteikšanu par 2\%, tātad klientu apkalpošanas speciālistiem bija jāveic izmaiņas tikai 8\% pieprasījumu rēķinu kategorijā \cite{paikens2020}.

Tipiski soļi nodomu noteikšanas pielietojumam uzņēmējdarbībā:
\begin{enumerate}
	\item Atrast visbiežākos pieprasījumu tipus;
	\item Sagatavot atbildes veidni (\textit{template});
	\item Nodomu noteikšanas sistēma identificē, vai lietotāja pieprasījums pieder iepriekš definētajaiem tipiem un izdod potenciālo atbildi;
	\item Klientu apkalpošanas speciālists izvērtē un koriģē atbildi pirms nosūtīšanas;
	\item Automātiski uzlabot nodomu noteikšanas sistēmu, balstoties uz speciālista veiktajām korekcijām \cite{paikens2020}.
\end{enumerate}



\section{Nodomu noteikšanas korpusi}


Viens no svarīgākajiem korpusiem tieši nodomu noteikšanā ir aviokompāniju ceļojumu informācijas sistēmu (ATIS -- \textit{Airline Travel Information Systems}) datu kopa. Tā ir audioierakstu un manuālu transkriptu datu kopa, kas sastāv no cilvēku sarunām ar automatizētām aviolīniju ceļojumu informācijas sistēmām. ATIS datu kopa nodrošina lielu ziņojumu un ar tiem saistīto nodomu skaitu, ko plaši izmanto kā novērtējuma (\textit{benchmark}) datu kopu klasifikatoru apmācībai nodomu noteikšanā \cite{atis1990}.


SNIPS -- \textit{Spoken Natural Language Interaction for Personal Assistant}) datu kopā ir 16 000 ievadi, kas sadalīti septiņos nodomos: SearchCreativeWork, GetWeather, BookRestaurant, PlayMusic, AddToPlaylist, RateBook, SearchScreeningEvent \cite{snips-2018}.


Lietotāja dabiskās valodas ievads tiek pārveidots jēdzientelpā, un tad pēc tam klasifikators nosaka lietotāja nodomu.

Dabisko valodu saprašanas (Natural Language Understanding (NLU)) uzdevumi iedalās divās daļās: nodomu noteikšana un parametru \textit{slots}) iegūšana no ievadiem \textit{slot filling}) (\ref{tab:slots} tabula) \cite{snips-2018}. Šajā darbā tiks apskatīts pirmais solis -- nodomu noteikšana.


\begin{table}[htbp]
    \centering
    \caption{``Weather" datu kopa, uz kuras tika apmācīts virtuālais asistents, kas apstrādā lietotāja vaicājumus par laikapstākļiem  \cite{snips-2018}}
    \begin{tabular}{rlr}
        Nodoms            & Parametrs                                           & Piemēri                                                                                                                                            \\
        & \textcolor[rgb]{ .439,  .188,  .627}{ region}           & Is it \textcolor[rgb]{ 1,  .753,  0}{cloudy} in \textcolor[rgb]{ 1,  0,  0}{Germany} \textcolor[rgb]{ .573,  .816,  .314}{ right now}?                         \\
        & \textcolor[rgb]{ 1,  0,  0}{ country}                   & Is \textcolor[rgb]{ .439,  .188,  .627}{South Carolina} expected to be \textcolor[rgb]{ 1,  .753,  0}{sunny} \textcolor[rgb]{ .573,  .816,  .314}{in 2 hours}? \\
        ForecastCondition & \textcolor[rgb]{ .573,  .816,  .314}{ datetime}         & Is there \textcolor[rgb]{ 1,  .753,  0}{snow} in \textcolor[rgb]{ 0,  .69,  .941}{Paris}?                                                                  \\
        & \textcolor[rgb]{ 0,  .69,  .941}{ locality}             & Should I expect a \textcolor[rgb]{ 1,  .753,  0}{storm} near \textcolor[rgb]{ .929,  .49,  .192}{Mount Rushmore}?                                          \\
        & \textcolor[rgb]{ 1,  .753,  0}{ condition}              &                                                                                                                                                    \\
        & \textcolor[rgb]{ .929,  .49,  .192}{ point of interest} &                                                                                                                                                    \\
    \end{tabular}%
    \label{tab:slots}%
\end{table}%


Korpusi ir paredzēti izmantošanai dabiskās valodas izpratnes uzdevumiem -- nodomu noteikšanai un entītiju atpazīšanai.

Ask Ubuntu korpuss sastāv no 162 jautājumiem un atbildēm, kas iegūti vietnē \url{https://askubuntu.com}. Šis korpuss ir anotēts ar pieciem dažādiem nolūkiem: MakeUpdate, SetupPrinter, ShutdownComputer, SoftwareRecommendation un None. Papildus šiem nolūkiem korpusā ir iekļauti arī trīs entītiju veidi: printeris, programmatūra un versija, tomēr entītiju noteikšana ir ārpus šī darba tvēruma \textit{scope}) \cite{braun-2017}. Pateicoties konkrētiem nodomiem un entītiju veidiem, izstrādātāji var koncentrēties uz precīzāku modeļu izveidi konkrētiem uzdevumiem, piemēram, printera iestatīšanai vai programmatūras atjaunināšanai.

Ask Ubuntu korpuss ir noderīgs resurss Ubuntu un citu Linux operētājsistēmu lietotāju pieredzes uzlabošanai. Tā kā Ubuntu ir pieejams daudzās valodās virutālā asistenta spēja atbildēt uz Ubuntu jautājumiem lietotāja izvēlētajā valodā var ievērojami palielināt operētājsistēmas pieejamību un lietojamību. Piemēram, trenējot virtuālos asistentus uz šī korpusa izstrādātāji var izveidot lietotājiem draudzīgas saskarnes \textit{interfaces}). 


Tīmekļa lietojumprogrammu korpuss (\textit{Web Applications Corpus}), turpmāk "webapps" korpuss ir vērtīgs resurss virtuālo asistentu izstrādei, kas var palīdzēt lietotājiem orientēties un novērst ar tīmekļa lietojumprogrammām saistītas problēmas. Ar 89 jautājumiem un atbildēm, kas iegūtas vietnē \url{https://webapps.stackexchange.com}, šis korpuss aptver dažādas tēmas un scenārijus, ar kuriem lietotāji var saskarties, izmantojot tīmekļa lietojumprogrammas. Korpusam ir astoņas anotācijas -- ChangePassword, DeleteAccount, DownloadVideo, ExportData, FilterSpam, FindAlternative, SyncAccounts un None --, kā arī trīs entītiju veidi: WebService, OS un Browser \cite{braun-2017}. Uz šī korpusa apmācīti mašīnmācīšanās modeļi var noteikt lietotāju nodomus un sniegt atbildes, tādā veidā uzlabojot lietotāja pieredzi ar tīmekļa lietojumprogrammām.

Chatbot korpuss tika iegūts no Telegram virtuālā asistenta Minhenes sabiedriskajam transportam. Tas satur 206 jautājumus, kas tika anotēti divos nolūkos -- DepartureTime un FindConnection --, kā arī piecus entītiju veidus: StationStart, StationDest, Criterion, Vehicle un Line \cite{braun-2017}. Šis korpuss ir īpaši noderīgs, lai izstrādātu virtuālos asistentus, kas palīdz lietotājiem orientēties sabiedriskajā transportā, jo tas nodrošina skaidru un kodolīgu jautājumu un entītiju kopumu šajā jomā. 

Askubuntu, Webapps un Chatbot korpusi tika izveidoti, lai novērtētu komerciāli pieejamo NLU servisu precizitāti Braun et al. rakstā \cite{braun-2017}, un ir pieejami \url{https://github.com/sebischair/NLU-Evaluation-Corpora}. Darbā tika izmantots uz šī korpusa balstīta datu kopa, kas tika izmantota nodomu noteikšanai \cite{fasttext2019} rakstā un ir pieejama \url{https://github.com/tilde-nlp/NLU-datasets}.

Jaunievedumi Balodis et al datu kopā:
\begin{itemize}
    \item iztulkota latviešu, krievu, igauņu un lietuviešu valodās
    \item attīrīta no entītijām, atbildēm uz lietotāju jautājumiem un lietotājvārdiem
\end{itemize}