
Nodomu noteikšanas sistēma identificē lietotāja brīvā valodā rakstīta piepasījuma tipu.

Jāpiebilst, ka labuma gūšanai no automatizācijas nav nepieciešams pārklāt 100\% lietotāju pieprasījumu. Veiksmīgas izmantošanas piemērs telekomunikāciju industrijā validācijā izmantoja 1732 klientu pieprasījumu datu kopu anotētu ar attiecīgajiem nolūkiem. Šajā gadījumā divi visbiežākie nodomi ir rēķina atlikšana (356 pieprasījumi; 21\% datu kopas) un nokavēta rēķina maksājuma apstiprināšana (207 pieprasījumi; 12\% datu kopas). Trīs mēnešus ilgā eksperimentālā pētījuma tika apstrādāti 14000 lietotāju pieprasījumi. Sākotnējos testos nodomu noteikšana un izvēlētā atbildes veidne bija precīza 90\% gadījumu, eksperimenta gaitā iegūtie dati ļāva uzlabot nodomu noteikšanu par 2\%, tātad klientu apkalpošanas speciālistiem bija jāveic izmaiņas tikai 8\% pieprasījumu rēķinu kategorijā \cite{paikens2020}.

Tipiski soļi nodomu noteikšanas biznesa pielietojumā:
\begin{enumerate}
	\item Atrast visbiežākos pieprasījumu tipus;
	\item Sagatavot atbildes veidni (template);
	\item Nodomu noteikšanas sistēma identificē, vai lietotāja pieprasījums pieder iepriekšdefinētajaiem tipiem un izdod potenciālo atbildi;
	\item Klientu apkalpošanas speciālists izvērtē un koriģē atbildi pirms nosūtīšanas;
	\item Automātiski uzlabot nodomu noteikšanas sistēmu, balstoties uz speciālista veiktajām korekcijām \cite{paikens2020}.
\end{enumerate}
