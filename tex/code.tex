Koda piemērs literatūras ievadā.
Krāsu dati "xkcd.json"
\url{https://github.com/dariusk/corpora/blob/master/data/colors/xkcd.json}.\\
Ideja un hex\_to\_int un closest funkcijas \cite{parrish2017}, pārējās pārrakstītas ātrdarbībai ar numpy.

\begin{python}
import numpy as np
import json

def hex_to_int(s):
    s = s.lstrip("#")
    return int(s[:2], 16), int(s[2:4], 16), int(s[4:6], 16)


def distance(coord1, coord2):
	"""Euclidean distance between two points
	"""
	return np.sqrt(np.sum(np.subtract(coord1, coord2)**2))


def subtractv(coord1, coord2):
	"""coord1 - coord2
	"""
	return np.subtract(coord1, coord2)


def addv(coord1, coord2):
	"""coord1 + coord2
	"""
	return np.sum([coord1, coord2], axis=0)


def closest(space, coord, n=10):
	closest = []
	for key in sorted(space.keys(),
						key=lambda x: distance(coord, space[x]))[:n]:
		closest.append(key)
	return closest


color_data = json.loads(open("xkcd.json").read())

colors = dict()
for item in color_data['colors']:
    colors[item["color"]] = hex_to_int(item["hex"])

\end{python}