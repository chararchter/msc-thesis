NLP (natural language processing) - dabisko valodu apstrāde.\\
Jēdzientelpa (word embeddings) - vārdu vai frāžu attēlojums daudzdimensionālā vektoru telpā.\\
Word2Vec (word to vector) - jēdzientelpas implementācija, kurā individuālus vārdus aizstāj daudzimensionāli vektori.\\
PCA (Principal Component Analysis) - galveno komponentu analīze.\\
GPT (Generative Pre-trained Transformer) – dziļo neironu tīklu modelis, kas spēj producēt tekstu, kas līdzīgs cilvēka rakstītam.\\
Transformeris (transformer) - dziļās mācīšanās modelis ar uzmanības (attention) mehānismu, kas spēj novērtēt ievades daļas nozīmīgumu.\\
Pārpielāgošana (overfitting) – pārmērīga pielāgošanās kādam konkrētai datu kopai, zaudējot spēju ģeneralizēt uz citām datu kopām.\\
Pietrenēšana (fine-tuning) - metode, kurā iepriekš apmācīts modelis tiek pietrenēts jaunam uzdevumam.