Multilingual user intent recognition is essential in the operation of virtual assistants, and automated customer service becomes increasingly cost-effective and relevant. User intents are determined by mapping input text strings to a multidimensional vector space or word embeddings. Then based on the word embedding a machine learning model classifies the intent to deliver the necessary information to users.

This work uses an annotated corpus for intent determination, containing user input and intent pairs in English, Latvian, Russian, Estonian, and Lithuanian. The study compares the accuracy of intent detection for multilingual word embeddings generated by mBERT and XLM-RoBERTa models, as well as three different intent detection approaches for each language (in the original language and machine translated into English), testing the intent classification model trained on:
%the same language corpus
%the corpus of all five languages
%only the English language corpus.

\begin{itemize}
    \item the same language corpus;
    \item the corpus of all five languages;
    \item only the English language corpus.
\end{itemize}

The results indicate that multilingual word embeddings and training on multilingual corpora can improve intent detection accuracy, but results may vary depending on the language.


\keywords{multilingual word embeddings, intent detection, mBERT, XLM-RoBERTa}
