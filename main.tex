% Compile with: latexmk -pdfxe -outdir=build
\documentclass[12pt,oneside]{LuThesis}
\usepackage[dvipsnames]{xcolor}
\definecolor{bostonuniversityred}{rgb}{0.8, 0.0, 0.0}
\definecolor{awesome}{rgb}{1.0, 0.13, 0.32}
\definecolor{cadmiumgreen}{rgb}{0.0, 0.42, 0.24}
\usepackage{listings}
\lstset{language=R,%
    %basicstyle=\color{red},
    basicstyle=\ttfamily,%
    breaklines=true,%
    morekeywords={matlab2tikz},
    keywordstyle=\color{blue},%
    morekeywords=[2]{1}, keywordstyle=[2]{\color{black}},
    showstringspaces=false,%without this there will be a symbol in the places where there is a space
    numbers=left,%
    numberstyle={\tiny \color{black}},% size of the numbers
    numbersep=9pt, % this defines how far the numbers are from the text
    emph=[1]{for,in,end,break},emphstyle=[1]\color{bostonuniversityred}, %some words to emphasise
    %emph=[2]{word1,word2}, emphstyle=[2]{style}, 
    emph=[1]{data.frame},emphstyle=[1]\color{awesome}, %some words to emphasise   
}

\title{Daudzvalodīgu jēdzientelpu pielietojums nodomu noteikšanā}
\author{Viktorija Leimane}

%%% LuThesis deklerācijas
\def\studaplnum{vl16047}
\def\darbvad{Dr.sc.comp. Kaspars Balodis}
\def\dokfak{\textbf{Datorikas fakultātē}}
\def\recenzents{Recenzents}
\def\doktitle{\textbf{Daudzvalodīgu jēdzientelpu pielietojums nodomu noteikšanā}}
\def\dokdate{} % Darba vadītāja parakstīšanas datums
\def\dokdateiesn{} % Darba iesniegšanas datums

%%% Valodas un fonti
\setdefaultlanguage{latvian}
\setotherlanguage{english}
\setmainfont[BoldFont=FreeSerifBold.ttf,ItalicFont=FreeSerifItalic.ttf,BoldItalicFont=FreeSerifBoldItalic.ttf]{FreeSerif.ttf}

\usepackage{graphicx}

%%% Misc
% \usepackage{hyperref}
\usepackage[hidelinks]{hyperref}
\addbibresource{thesis.bib} % biblatex bibliotēkas fails: thesis.bib
\usepackage[section]{placeins}
\usepackage[toc,page]{appendix}

%%% Dokumenta struktūra
\begin{document}

\maketitle

% * Anotācija
\chapter*{Anotācija}
\setcounter{page}{1}
\begin{abstract}
Daudzvalodīga lietotāja nodomu noteikšana ir nozīmīga virtuālo asistentu darbībā, un klientu apkalpošanas automatizācija kļūst arvien izdevīgāka un aktuālāka. Viens veids noteikt nodomu ir attēlojot ievades teksta virknes daudzdimensionālā vektoru telpā jeb jēdzientelpā, kuru izmanto nodomu klasifikācijas modeļi, lai piegādātu lietotājiem tiem nepieciešamo informāciju. Darbā tiks apmācīti dažādi mašīnmācīšanās modeļi un salīdzinātas dažādas pieejas: ievades teksta attēlojums uz daudzvalodīgu tekstu korpusu apmācītas jēdzientelpas un ievades teksta mašīntulkošana uz angļu valodu un attēlojums uz angļu valodas korpusa apmācītas jēdzientelpas.

\keywords{daudzvalodīgas jēdzientelpas, nodomu noteikšana}

\end{abstract}

\chapter*{Abstract}
\begin{english}
\begin{abstract}
Multilingual user intent recognition is essential in the operation of virtual assistants, and automated customer service becomes increasingly cost-effective and relevant. User intents are determined by mapping input text strings to a multidimensional vector space or word embeddings. Then based on the word embedding a machine learning model classifies the intent to deliver the necessary information to users.

This work uses an annotated corpus for intent determination, containing user input and intent pairs in English, Latvian, Russian, Estonian, and Lithuanian. The study compares the accuracy of intent detection for multilingual word embeddings generated by mBERT and XLM-RoBERTa models, as well as three different intent detection approaches for each language (in the original language and machine translated into English), testing the intent classification model trained on:
%the same language corpus
%the corpus of all five languages
%only the English language corpus.

\begin{itemize}
    \item the same language corpus;
    \item the corpus of all five languages;
    \item only the English language corpus.
\end{itemize}

The results indicate that multilingual word embeddings and training on multilingual corpora can improve intent detection accuracy, but results may vary depending on the language.


\keywords{multilingual word embeddings, intent detection, mBERT, XLM-RoBERTa}

\end{abstract}
\end{english}

%* Saturs
\tableofcontents


\chapter*{Apzīmējumu saraksts}
\addcontentsline{toc}{chapter}{Apzīmējumu saraksts}
NLP (\textit{natural language processing}) -- dabisko valodu apstrāde.\\
Jēdzientelpa (\textit{word embeddings}) -- vārdu vai frāžu attēlojums daudzdimensionālā vektoru telpā.\\
Word2Vec (\textit{word to vector}) -- jēdzientelpas implementācija, kurā individuālus vārdus aizstāj daudzimensionāli vektori.\\
PCA (\textit{Principal Component Analysis}) -- galveno komponentu analīze.\\
GPT (\textit{Generative Pre-trained Transformer}) – dziļo neironu tīklu modelis, kas spēj producēt tekstu, kas līdzīgs cilvēka rakstītam.\\
Transformeris (\textit{transformer}) -- dziļās mācīšanās modelis ar uzmanības (attention) mehānismu, kas spēj novērtēt ievades daļas nozīmīgumu.\\
Pārpielāgošana (\textit{overfitting}) -- pārmērīga pielāgošanās kādam konkrētai datu kopai, zaudējot spēju ģeneralizēt uz citām datu kopām.\\
Pietrenēšana (\textit{fine-tuning}) -- metode, kurā iepriekš apmācīts modelis tiek pietrenēts jaunam uzdevumam.


\chapter*{Ievads}
\addcontentsline{toc}{chapter}{Ievads}
Dabiskā valodas apstrāde (NLP -- \textit{natural language processing}) ir starpdisciplināra datorlingvistikas un mākslīgā intelekta nozare, kas strādā pie tā, lai datori varētu saprast cilvēka dabiskās valodas ievadi. Dabiskās valodas pēc būtības ir sarežģītas, un daudzi NLP uzdevumi ir slikti piemēroti matemātiski precīziem algoritmiskajiem risinājumiem. Palielinoties korpusu (liela apjoma rakstītas vai runātas dabiskās valodas kolekcija) pieejamībai, NLP uzdevumi arvien biežāk un efektīvāk tiek risināti ar mašīnmācīšanās modeļiem \cite{nlp2018}.


Arvien lielāku daļu tirgus pārņem pakalpojumu industrija,
%(\% ES)
un pakalpojumi arvien biežāk tiek piedāvāti starptautiski. Tam ir nepieciešams lietotāju dzimtās valodas atbalsts gan valstu valodu regulējumu, gan tirgus nišas ieņemšanas un tirgus konkurences dēļ.
%(\% ES iedzīvotāju svarīgi saņemt pakalpojumu savā dzimtajā valodā). 

Dabiskās valodas apstrādei ir liels biznesa potenciāls, jo tas ļauj uzņēmumiem palielināt peļņu samazinot izdevumus, no kuriem lielākais parasti ir darbs. %\% pakalpojumu nozares uzņēmumu izdevumi ir darbs. Var minēt tech layoffs, specifically, Meta’s Year of Efficiency % original sentence: Uzņēmumiem tas ir izdevīgi, jo ļauj samazināt personālizdevumus.
% (\% pakalpojumu nozares uzņēmuma izdevumu)
Tas savukārt samazina barjeru iekļūšanai un dalībai starptautiskā tirgū, kas nozīmē lielāku piedāvāto pakalpojumu daudzveidību un konkurenci, tātad zemākas izmaksas patērētājam. Lietotājiem, kuru dzimto valodu pārvalda mazs cilvēku skaits kā tas ir, piemēram, latviešu valodā, kļūst pieejami pakalpojumi, kuru tulkojumus būtu ekonomiski nerentabli nodrošināt ar algotu profesionālu personālu.

Darbā apskatītā metode nodrošina automatizāciju divos veidos: 
\begin{itemize}
	\item daudzvalodīgs modelis aizvieto profesionālu tulkotāju;
	\item virtuālais asistents aizvieto klientu apkalpošanas speciālistu.
\end{itemize}

Darbs ir sadalīts teorētiskajā un praktiskajā daļā. Teoretiskajā daļā ir īsi aprakstīti mūsdienu modeļi un pieejas. Praktiskajā daļā ir veikti eksperimenti ar mērķi pielietot daudzvalodīgus modeļus un salīdzināt tos ar esošiem risinājumiem.

Pētījuma jautājums: Kādas ir efektīvākās metodes un daudzvalodīgi jēdzientelpu modeļi daudzvalodu nodomu noteikšanai?


\chapter{Literatūras apskats}
% \section{Definīcijas}
\noindent \textbf{Saules plankums} - magnētiskās plūsmas koncentrācija bipolāros klāsteros vai grupās, kas novērojama kā tumšs plankums uz Saules fotosfēras.\\
\textbf{Saules plankumu cikls} - aptuveni 11 gadus ilga kvaziperiodiska variācija saules plankuma skaitlī. Magnētiskā lauka polaritātes modelis mainās ar katru ciklu.\\
\textbf{Saules plankuma skaitlis} - Dienas saules plankuma aktivitātes indekss (R), definēts kā $R = k(10 \cdot g + s)$, kur
s - individuālo plankumu skaits;
g - saules plankumu grupu skaits;
k - observatorijas faktors.\\
\textbf{SSI} - saules spektrālais starojums vai spektra enerģijas blīvums - saules jaudas izkliede uz virsmas laukuma vienību.\\
% total solar irradiance (TSI) - Solar energy per unit time over a unit area perpendicular to the Sun’s rays at the top of Earth’s atmosphere.
\textbf{TSI} - Saules starojuma absolūtās intensitātes mērījums integrēts visā saules enerģijas diskā un visā saules enerģijas spektrā.\\
% Laboratory for Atmospheric and Space Physics, University of Colorado (2019)
% \\http://lasp.colorado.edu/home/sorce/reference/glossary/
\textbf{Izstarojums} - starojuma avota jaudas incidents uz virsmas laukuma vienību.\\ %Irradiance
\textbf{Saules diennakts kustība} (diurnal motion) - Debess spīdekļu redzamā pārvietošanās pie debess sfēras (rotācija ap pasaules asi) diennakts laikā.\\ %Diurnal motion
% Tās faktiskais cēļonis ir Zemes rotācija ap asi. Diennakts kurstībā visi debess spīdekļi pārvietojas pa debess paralēlēm.
Beam Radiation - the solar radiation received from the sun without having been scattered by the atmosphere (also known as direct solar radiation)
Diffuse Radiation - the solar radiation received from the sun after its direction has been changed by scattering by the atmosphere
\input{tex/literature}


\chapter{Rezultāti}
TODO: apraksti grafikiem un tabula līdz ko būs final version.

\section{Chatbot datu kopa}

\begin{table}[htbp]
  \centering
  \caption{Unikālo nodomu skaits "chatbot" treniņkopā un testa kopā.}
    \begin{tabular}{lrrr} \toprule
    Nodoms & Treniņkopā & Testa kopā & $\Sigma$ \\\midrule
    FindConnection & 57    & 71 & 128 \\
    DepartureTime & 43    & 35 & 78 \\
    $\Sigma$ & 100    & 106 & 206 \\\bottomrule
    \end{tabular}%
  \label{tab:chatbot-labels}%
\end{table}%


% Table generated by Excel2LaTeX from sheet 'chatbot_bert-base-multilingual-'
\begin{table}[htbp]
  \centering
  \caption{Nodomu noteikšanas precizitāte uz \textit{Chatbot} datukopas ar mBERT modeli, \%}
  % \caption{mBERT rezultāti}
    \begin{tabular}{lrrrrrr} \toprule
    languages & 1a & 1b & 2a & 2b & 3a & 3b \\\midrule
    % languages & 1\_translated & 1\_untranslated & 2\_translated & 2\_untranslated & 3\_translated & 3\_untranslated \\\midrule
    en    &   --  & 91.51 &  --   & 90.57 &  --   & 95.28 \\
    lv    & 90.57 & 88.68 & 94.34 & 93.40 & 92.45 & 85.85 \\
    ru    & 89.62 & 92.45 & 93.40 & 91.51 & 89.62 & 93.40 \\
    et    & 89.62 & 87.74 & 86.79 & 88.68 & 90.57 & 58.49 \\
    lt    & 95.28 & 94.34 & 94.34 & 94.34 & 91.51 & 79.25 \\\bottomrule
    \end{tabular}%
  \label{tab:chatbot-bert}%
\end{table}%

% Table generated by Excel2LaTeX from sheet 'chatbot_bert-base-multilingual-'
\begin{table}[htbp]
  \centering
  \caption{Nodomu noteikšanas precizitāte uz \textit{Chatbot} datukopas ar mBERT modeli, \%}
  % \caption{mBERT rezultāti}
    \begin{tabular}{lrrrrrr} \toprule
    languages & 1a & 1b & 2a & 2b & 3a & 3b \\\midrule
    % languages & 1\_translated & 1\_untranslated & 2\_translated & 2\_untranslated & 3\_translated & 3\_untranslated \\\midrule
    en    &  --   & \cellcolor[rgb]{ .988,  .988,  1}91.51 &    -- & \cellcolor[rgb]{ .984,  .969,  .98}90.57 &  --   & \cellcolor[rgb]{ .353,  .541,  .776}95.28 \\
    lv    & \cellcolor[rgb]{ .984,  .969,  .98}90.57 & \cellcolor[rgb]{ .984,  .937,  .949}88.68 & \cellcolor[rgb]{ .514,  .655,  .835}94.34 & \cellcolor[rgb]{ .671,  .765,  .89}93.40 & \cellcolor[rgb]{ .831,  .878,  .945}92.45 & \cellcolor[rgb]{ .984,  .886,  .898}85.85 \\
    ru    & \cellcolor[rgb]{ .984,  .953,  .965}89.62 & \cellcolor[rgb]{ .831,  .878,  .945}92.45 & \cellcolor[rgb]{ .671,  .765,  .89}93.40 & \cellcolor[rgb]{ .988,  .988,  1}91.51 & \cellcolor[rgb]{ .984,  .953,  .965}89.62 & \cellcolor[rgb]{ .671,  .765,  .89}93.40 \\
    et    & \cellcolor[rgb]{ .984,  .953,  .965}89.62 & \cellcolor[rgb]{ .984,  .922,  .933}87.74 & \cellcolor[rgb]{ .984,  .906,  .914}86.79 & \cellcolor[rgb]{ .984,  .937,  .949}88.68 & \cellcolor[rgb]{ .984,  .969,  .98}90.57 & \cellcolor[rgb]{ .973,  .412,  .42}58.49 \\
    lt    & \cellcolor[rgb]{ .353,  .541,  .776}95.28 & \cellcolor[rgb]{ .514,  .655,  .835}94.34 & \cellcolor[rgb]{ .514,  .655,  .835}94.34 & \cellcolor[rgb]{ .514,  .655,  .835}94.34 & \cellcolor[rgb]{ .988,  .988,  1}91.51 & \cellcolor[rgb]{ .98,  .773,  .784}79.25 \\\bottomrule
    \end{tabular}%
  \label{tab:chatbot-bert}%
\end{table}%


% Table generated by Excel2LaTeX from sheet 'chatbot_xlm-roberta-base_result'
\begin{table}[htbp]
  \centering
  \caption{Nodomu noteikšanas precizitāte uz \textit{Chatbot} datukopas ar XLM-RoBERTa modeli, \%}
  % \caption{Add caption}
    \begin{tabular}{lrrrrrr} \toprule
    languages & 1a & 1b & 2a & 2b & 3a & 3b \\\midrule
    % languages & 1\_translated & 1\_untranslated & 2\_translated & 2\_untranslated & 3\_translated & 3\_untranslated \\\midrule
    en    &   --  & 95.28 &   --  & 95.28 &  --   & 96.23 \\
    lv    & 91.51 & 88.68 & 93.40 & 93.40 & 94.34 & 85.85 \\
    ru    & 92.45 & 94.34 & 88.68 & 94.34 & 91.51 & 90.57 \\
    et    & 88.68 & 91.51 & 88.68 & 89.62 & 91.51 & 63.21 \\
    lt    & 92.45 & 91.51 & 91.51 & 93.40 & 90.57 & 77.36 \\\bottomrule
    \end{tabular}%
  \label{tab:chatbot-xlm}%
\end{table}%


% Table generated by Excel2LaTeX from sheet 'chatbot_xlm-roberta-base_result'
\begin{table}[htbp]
  \centering
  \caption{Nodomu noteikšanas precizitāte uz \textit{Chatbot} datukopas ar XLM-RoBERTa modeli, \%}
    \begin{tabular}{lrrrrrr} \toprule
    languages & 1a & 1b & 2a & 2b & 3a & 3b \\\midrule
    % languages & 1\_translated & 1\_untranslated & 2\_translated & 2\_untranslated & 3\_translated & 3\_untranslated \\\midrule
    en    &   --  & \cellcolor[rgb]{ .482,  .631,  .824}95.28 &  --   & \cellcolor[rgb]{ .482,  .631,  .824}95.28 &  --   & \cellcolor[rgb]{ .353,  .541,  .776}96.23 \\
    lv    & \cellcolor[rgb]{ .988,  .988,  1}91.51 & \cellcolor[rgb]{ .984,  .929,  .941}88.68 & \cellcolor[rgb]{ .737,  .812,  .914}93.40 & \cellcolor[rgb]{ .737,  .812,  .914}93.40 & \cellcolor[rgb]{ .608,  .722,  .867}94.34 & \cellcolor[rgb]{ .984,  .871,  .882}85.85 \\
    ru    & \cellcolor[rgb]{ .863,  .902,  .957}92.45 & \cellcolor[rgb]{ .608,  .722,  .867}94.34 & \cellcolor[rgb]{ .984,  .929,  .941}88.68 & \cellcolor[rgb]{ .608,  .722,  .867}94.34 & \cellcolor[rgb]{ .988,  .988,  1}91.51 & \cellcolor[rgb]{ .984,  .969,  .98}90.57 \\
    et    & \cellcolor[rgb]{ .984,  .929,  .941}88.68 & \cellcolor[rgb]{ .988,  .988,  1}91.51 & \cellcolor[rgb]{ .984,  .929,  .941}88.68 & \cellcolor[rgb]{ .984,  .949,  .961}89.62 & \cellcolor[rgb]{ .988,  .988,  1}91.51 & \cellcolor[rgb]{ .973,  .412,  .42}63.21 \\
    lt    & \cellcolor[rgb]{ .863,  .902,  .957}92.45 & \cellcolor[rgb]{ .988,  .988,  1}91.51 & \cellcolor[rgb]{ .988,  .988,  1}91.51 & \cellcolor[rgb]{ .737,  .812,  .914}93.40 & \cellcolor[rgb]{ .984,  .969,  .98}90.57 & \cellcolor[rgb]{ .98,  .698,  .71}77.36 \\\bottomrule
    \end{tabular}%
  \label{tab:chatbot-xlm}%
\end{table}%




\begin{figure}[h] 
   \centering
   \subcaptionbox{mBERT latviešu treniņdatu kopa}{\includegraphics[width=0.49\linewidth,trim={0 0.1cm 0 0},clip]{graphs/bert-base-multilingual-cased_lv-accuracy.png}}
   \subcaptionbox{mBERT mašīntulkoto latviešu treniņdatu kopa}{\includegraphics[width=0.49\linewidth,trim={0 0.1cm 0 0},clip]{graphs/bert-base-multilingual-cased_lv_en-accuracy.png}}
   \caption{caption} 
   \label{fig:chatbot-bert}
\end{figure}


\begin{figure}[h] 
   \centering
   \subcaptionbox{mBERT apvienotā treniņdatu kopa}{\includegraphics[width=0.49\linewidth,trim={0 0.1cm 0 0},clip]{graphs/bert-base-multilingual-cased_all-accuracy.png}}
   \subcaptionbox{mBERT apvienotā mašīntulkoto treniņdatu kopa}{\includegraphics[width=0.49\linewidth,trim={0 0.1cm 0 0},clip]{graphs/bert-base-multilingual-cased_all_en-accuracy.png}}
   \caption{caption} 
   \label{fig:chatbot-bert-all}
\end{figure}


\begin{figure}[h] 
   \centering
   \subcaptionbox{mBERT angļu treniņdatu kopa}{\includegraphics[width=0.49\linewidth,trim={0 0.1cm 0 0},clip]{graphs/bert-base-multilingual-cased_en-accuracy.png}}
   \subcaptionbox{XLM-RoBERTa angļu treniņdatu kopa}{\includegraphics[width=0.49\linewidth,trim={0 0.1cm 0 0},clip]{graphs/xlm-roberta-base_en-accuracy.png}}
   \caption{caption} 
   \label{fig:chabot-bert-xlm-en}
\end{figure}


\begin{figure}[h] 
   \centering
   \subcaptionbox{XLM-RoBERTa latviešu treniņdatu kopa}{\includegraphics[width=0.49\linewidth,trim={0 0.1cm 0 0},clip]{graphs/xlm-roberta-base_lv-accuracy.png}}
   \subcaptionbox{XLM-RoBERTa mašīntulkoto latviešu treniņdatu kopa}{\includegraphics[width=0.49\linewidth,trim={0 0.1cm 0 0},clip]{graphs/xlm-roberta-base_lv_en-accuracy.png}}
   \caption{caption} 
   \label{fig:chatbot-xlm}
\end{figure}


\begin{figure}[h] 
   \centering
   \subcaptionbox{XLM-RoBERTa apvienotā treniņdatu kopa}{\includegraphics[width=0.49\linewidth,trim={0 0.1cm 0 0},clip]{graphs/xlm-roberta-base_all-accuracy.png}}
   \subcaptionbox{XLM-RoBERTa apvienotā mašīntulkoto treniņdatu kopa}{\includegraphics[width=0.49\linewidth,trim={0 0.1cm 0 0},clip]{graphs/xlm-roberta-base_all_en-accuracy.png}}
   \caption{caption} 
   \label{fig:chatbot-xlm-all}
\end{figure}


\section{Askubuntu datu kopa}

\begin{table}[htbp]
  \centering
  \caption{Unikālo nodomu skaits "askubuntu" treniņkopā un testa kopā.}
    \begin{tabular}{lrrr} \toprule
    Nodoms & Treniņkopā & Testa kopā & $\Sigma$ \\\midrule
    Software Recommendation & 17    & 40 & 57 \\
    Make Update & 10    & 37 & 47 \\
    Shutdown Computer & 13    & 14 & 27 \\
    Setup Printer & 10    & 13 & 23\\
    None  & 3     & 5 & 8\\
   $\Sigma$ & 53    & 109 & 162 \\\bottomrule
    \end{tabular}%
  \label{tab:askubuntu-labels}%
\end{table}%


% Table generated by Excel2LaTeX from sheet 'askubuntu_bert-base-multilingua'
\begin{table}[htbp]
  \centering
  \caption{Nodomu noteikšanas precizitāte uz \textit{Askubuntu} datukopas ar mBERT modeli, \%}
    \begin{tabular}{lrrrrrr} \toprule
    languages & 1a & 1b & 2a & 2b & 3a & 3b \\\midrule
    % languages & 1\_translated & 1\_untranslated & 2\_translated & 2\_untranslated & 3\_translated & 3\_untranslated \\\midrule
    en    &  --   & \cellcolor[rgb]{ .984,  .965,  .976}78.90 &   --  & \cellcolor[rgb]{ .737,  .812,  .914}81.65 &  --   & \cellcolor[rgb]{ .608,  .722,  .867}82.57 \\
    lv    & \cellcolor[rgb]{ .984,  .965,  .976}78.90 & \cellcolor[rgb]{ .737,  .812,  .914}81.65 & \cellcolor[rgb]{ .988,  .988,  1}79.82 & \cellcolor[rgb]{ .353,  .541,  .776}84.40 & \cellcolor[rgb]{ .863,  .902,  .957}80.73 & \cellcolor[rgb]{ .973,  .412,  .42}54.13 \\
    ru    & \cellcolor[rgb]{ .482,  .631,  .824}83.49 & \cellcolor[rgb]{ .737,  .812,  .914}81.65 & \cellcolor[rgb]{ .984,  .882,  .894}75.23 & \cellcolor[rgb]{ .737,  .812,  .914}81.65 & \cellcolor[rgb]{ .988,  .988,  1}79.82 & \cellcolor[rgb]{ .976,  .655,  .667}65.14 \\
    et    & \cellcolor[rgb]{ .984,  .965,  .976}78.90 & \cellcolor[rgb]{ .482,  .631,  .824}83.49 & \cellcolor[rgb]{ .737,  .812,  .914}81.65 & \cellcolor[rgb]{ .984,  .925,  .937}77.06 & \cellcolor[rgb]{ .984,  .965,  .976}78.90 & \cellcolor[rgb]{ .973,  .494,  .502}57.80 \\
    lt    & \cellcolor[rgb]{ .608,  .722,  .867}82.57 & \cellcolor[rgb]{ .984,  .945,  .957}77.98 & \cellcolor[rgb]{ .984,  .965,  .976}78.90 & \cellcolor[rgb]{ .98,  .824,  .831}72.48 & \cellcolor[rgb]{ .863,  .902,  .957}80.73 & \cellcolor[rgb]{ .973,  .533,  .541}59.63 \\\bottomrule
    \end{tabular}%
  \label{tab:askubuntu-bert}%
\end{table}%

% uncolored just in case
% Table generated by Excel2LaTeX from sheet 'askubuntu_bert-base-multilingua'
\begin{table}[htbp]
  \centering
    \caption{Nodomu noteikšanas precizitāte uz \textit{Askubuntu} datukopas ar mBERT modeli, \%}
    \begin{tabular}{lrrrrrr} \toprule
    languages & 1a & 1b & 2a & 2b & 3a & 3b \\\midrule
    % languages & 1\_translated & 1\_untranslated & 2\_translated & 2\_untranslated & 3\_translated & 3\_untranslated \\\midrule
    en    &   --  & 78.90 &   --  & 81.65 &  --   & 82.57 \\
    lv    & 78.90 & 81.65 & 79.82 & 84.40 & 80.73 & 54.13 \\
    ru    & 83.49 & 81.65 & 75.23 & 81.65 & 79.82 & 65.14 \\
    et    & 78.90 & 83.49 & 81.65 & 77.06 & 78.90 & 57.80 \\
    lt    & 82.57 & 77.98 & 78.90 & 72.48 & 80.73 & 59.63 \\\bottomrule
    \end{tabular}%
  \label{tab:askubuntu-bert}%
\end{table}%


% Table generated by Excel2LaTeX from sheet 'askubuntu_xlm-roberta-base_resu'
\begin{table}[htbp]
  \centering
  \caption{Nodomu noteikšanas precizitāte uz \textit{Askubuntu} datukopas ar XLM-RoBERTa modeli, \%}
    \begin{tabular}{lrrrrrr}\toprule
    languages & 1a & 1b & 2a & 2b & 3a & 3b \\\midrule
    % languages & 1\_translated & 1\_untranslated & 2\_translated & 2\_untranslated & 3\_translated & 3\_untranslated \\\midrule
    en    &   --  & \cellcolor[rgb]{ .498,  .643,  .827}77.06 &  --   & \cellcolor[rgb]{ .353,  .541,  .776}78.90 &   --  & \cellcolor[rgb]{ .988,  .988,  1}70.64 \\
    lv    & \cellcolor[rgb]{ .984,  .922,  .933}67.89 & \cellcolor[rgb]{ .98,  .816,  .827}63.30 & \cellcolor[rgb]{ .353,  .541,  .776}78.90 & \cellcolor[rgb]{ .353,  .541,  .776}78.90 & \cellcolor[rgb]{ .639,  .741,  .878}75.23 & \cellcolor[rgb]{ .973,  .451,  .459}47.71 \\
    ru    & \cellcolor[rgb]{ .984,  .922,  .933}67.89 & \cellcolor[rgb]{ .78,  .843,  .929}73.39 & \cellcolor[rgb]{ .639,  .741,  .878}75.23 & \cellcolor[rgb]{ .639,  .741,  .878}75.23 & \cellcolor[rgb]{ .98,  .835,  .847}64.22 & \cellcolor[rgb]{ .976,  .561,  .569}52.29 \\
    et    & \cellcolor[rgb]{ .984,  .878,  .89}66.06 & \cellcolor[rgb]{ .984,  .965,  .976}69.72 & \cellcolor[rgb]{ .78,  .843,  .929}73.39 & \cellcolor[rgb]{ .353,  .541,  .776}78.90 & \cellcolor[rgb]{ .984,  .859,  .871}65.14 & \cellcolor[rgb]{ .973,  .494,  .502}49.54 \\
    lt    & \cellcolor[rgb]{ .984,  .965,  .976}69.72 & \cellcolor[rgb]{ .98,  .835,  .847}64.22 & \cellcolor[rgb]{ .427,  .592,  .804}77.98 & \cellcolor[rgb]{ .71,  .792,  .902}74.31 & \cellcolor[rgb]{ .988,  .988,  1}70.64 & \cellcolor[rgb]{ .973,  .412,  .42}45.87 \\
    \end{tabular}%
  \label{tab:askubuntu-xlm}%
\end{table}%




% Table generated by Excel2LaTeX from sheet 'askubuntu_xlm-roberta-base_resu'
\begin{table}[htbp]
  \centering
  \caption{Nodomu noteikšanas precizitāte uz \textit{Askubuntu} datukopas ar XLM-RoBERTa modeli, \%}
  % \caption{Askubuntu rezultāti ar XLM-RoBERTa modeli}
    \begin{tabular}{lrrrrrr}\toprule
    languages & 1a & 1b & 2a & 2b & 3a & 3b \\\midrule
    % languages & 1\_translated & 1\_untranslated & 2\_translated & 2\_untranslated & 3\_translated & 3\_untranslated \\\midrule
    en    &   --  & 77.06 &  --   & 78.90 &  --   & 70.64 \\
    lv    & 67.89 & 63.30 & 78.90 & 78.90 & 75.23 & 47.71 \\
    ru    & 67.89 & 73.39 & 75.23 & 75.23 & 64.22 & 52.29 \\
    et    & 66.06 & 69.72 & 73.39 & 78.90 & 65.14 & 49.54 \\
    lt    & 69.72 & 64.22 & 77.98 & 74.31 & 70.64 & 45.87 \\\bottomrule
    \end{tabular}%
  \label{tab:askubuntu-xlm}%
\end{table}%



\begin{figure}[h] 
   \centering
   \subcaptionbox{mBERT latviešu treniņdatu kopa}{\includegraphics[width=0.49\linewidth,trim={0 0.1cm 0 0},clip]{results-5/graphs/askubuntu_bert-base-multilingual-cased_lv-accuracy.png}}
   \subcaptionbox{mBERT mašīntulkoto latviešu treniņdatu kopa}{\includegraphics[width=0.49\linewidth,trim={0 0.1cm 0 0},clip]{results-5/graphs/askubuntu_bert-base-multilingual-cased_lv_en-accuracy.png}}
   \caption{caption} 
   \label{fig:askubuntu-bert}
\end{figure}


\begin{figure}[h] 
   \centering
   \subcaptionbox{mBERT apvienotā treniņdatu kopa}{\includegraphics[width=0.49\linewidth,trim={0 0.1cm 0 0},clip]{results-5/graphs/askubuntu_bert-base-multilingual-cased_all-accuracy.png}}
   \subcaptionbox{mBERT apvienotā mašīntulkoto treniņdatu kopa}{\includegraphics[width=0.49\linewidth,trim={0 0.1cm 0 0},clip]{results-5/graphs/askubuntu_bert-base-multilingual-cased_all_en-accuracy.png}}
   \caption{caption} 
   \label{fig:askubuntu-bert-all}
\end{figure}


\begin{figure}[h] 
   \centering
   \subcaptionbox{mBERT angļu treniņdatu kopa}{\includegraphics[width=0.49\linewidth,trim={0 0.1cm 0 0},clip]{results-5/graphs/askubuntu_bert-base-multilingual-cased_en-accuracy.png}}
   \subcaptionbox{XLM-RoBERTa angļu treniņdatu kopa}{\includegraphics[width=0.49\linewidth,trim={0 0.1cm 0 0},clip]{results-5/graphs/askubuntu_xlm-roberta-base_en-accuracy.png}}
   \caption{caption} 
   \label{fig:askubuntu-bert-xlm-en}
\end{figure}


\begin{figure}[h] 
   \centering
   \subcaptionbox{XLM-RoBERTa latviešu treniņdatu kopa}{\includegraphics[width=0.49\linewidth,trim={0 0.1cm 0 0},clip]{results-5/graphs/askubuntu_xlm-roberta-base_lv-accuracy.png}}
   \subcaptionbox{XLM-RoBERTa mašīntulkoto latviešu treniņdatu kopa}{\includegraphics[width=0.49\linewidth,trim={0 0.1cm 0 0},clip]{results-5/graphs/askubuntu_xlm-roberta-base_lv_en-accuracy.png}}
   \caption{caption} 
   \label{fig:askubuntu-xlm}
\end{figure}


\begin{figure}[h] 
   \centering
   \subcaptionbox{XLM-RoBERTa apvienotā treniņdatu kopa}{\includegraphics[width=0.49\linewidth,trim={0 0.1cm 0 0},clip]{results-5/graphs/askubuntu_xlm-roberta-base_all-accuracy.png}}
   \subcaptionbox{XLM-RoBERTa apvienotā mašīntulkoto treniņdatu kopa}{\includegraphics[width=0.49\linewidth,trim={0 0.1cm 0 0},clip]{results-5/graphs/askubuntu_xlm-roberta-base_all_en-accuracy.png}}
   \caption{caption} 
   \label{fig:askubuntu-xlm-all}
\end{figure}



\section{Webapps datu kopa}


Webapps datu kopā ir nodoms ar tikai vienu piemēru, kas izraisa "ValueError: The least populated class in y has only 1 member, which is too few. The minimum number of groups for any class cannot be less than 2." Tāpēc nodomi ar mazāk nekā trīs piemēriem tika apvienoti vienā nodomā "Other". Tas atbilst reālam pielietojumam industrijā, kur nodomi nav vienlīdzīgi pārstāvēti  -- piemēram, starp 115 dažādiem nodomiem divi visbiežākie nodomi kopā pārstāv 33\% datu kopas \cite{paikens2020} -- un ir svarīgi spēt atsijāt nodomus, kurus jāapstrādā klientu apkalpošanas speciālistam -- cilvēkam.



\begin{table}[htbp]
  \centering
  \caption{Unikālo nodomu skaits ``webapps" treniņkopā un testa kopā. Ar treknrakstu iezīmētas nodomi, kuri ir pietiekami pārstāvētas, pārējie nodomi tika apvienoti vienā jaunā nodomā: "Other"}
    \begin{tabular}{lrrr} \toprule
    Nodoms & Treniņkopā & Testa kopā & $\Sigma$ \\\midrule
    \textbf{Find Alternative} & \textbf{7} & \textbf{16} & 23\\
    \textbf{Delete Account} & \textbf{7} & \textbf{10} & 17\\
    \textbf{Filter Spam} & \textbf{6} & \textbf{14} & 20 \\
    \textbf{Sync Accounts} & \textbf{3} & \textbf{6} & 9 \\
    Change Password & 2     & 6 & 8\\
    None  & 2     & 4 & 6\\
    Export Data & 2     & 3 & 5 \\
    Download Video & 1     & 0 & 1\\
    $\Sigma$ & 30    & 59 & 89 \\\bottomrule
    \end{tabular}%
  \label{tab:webapps-labels}%
\end{table}%


% Table generated by Excel2LaTeX from sheet 'webapps_bert-base-multilingual-'
\begin{table}[htbp]
  \centering
  \caption{Nodomu noteikšanas precizitāte uz \textit{Webapps} datukopas ar mBERT modeli, \%}
  % \caption{Webapps rezultāti ar mBERT modeli}
    \begin{tabular}{lrrrrrr}\toprule
    languages & 1a & 1b & 2a & 2b & 3a & 3b \\\midrule
    % languages & 1\_translated & 1\_untranslated & 2\_translated & 2\_untranslated & 3\_translated & 3\_untranslated \\\midrule
    en    &   --  & 64.41 &   --  & 67.80 &  --   & 61.02 \\
    lv    & 67.80 & 57.63 & 64.41 & 66.10 & 69.49 & 30.51 \\
    ru    & 67.80 & 69.49 & 72.88 & 74.58 & 64.41 & 44.07 \\
    et    & 55.93 & 61.02 & 67.80 & 71.19 & 69.49 & 42.37 \\
    lt    & 62.71 & 49.15 & 67.80 & 66.10 & 66.10 & 28.81 \\\bottomrule
    \end{tabular}%
  \label{tab:webapps-bert}%
\end{table}%


% Table generated by Excel2LaTeX from sheet 'webapps_bert-base-multilingual-'
\begin{table}[htbp]
  \centering
  \caption{Nodomu noteikšanas precizitāte uz \textit{Webapps} datukopas ar mBERT modeli, \%}
    \begin{tabular}{lrrrrrr}\toprule
    languages & 1a & 1b & 2a & 2b & 3a & 3b \\\midrule
    % languages & 1\_translated & 1\_untranslated & 2\_translated & 2\_untranslated & 3\_translated & 3\_untranslated \\\midrule
    en    &   --  & \cellcolor[rgb]{ .984,  .961,  .973}64.41 &  --   & \cellcolor[rgb]{ .863,  .902,  .957}67.80 &   --  & \cellcolor[rgb]{ .984,  .906,  .918}61.02 \\
    lv    & \cellcolor[rgb]{ .863,  .902,  .957}67.80 & \cellcolor[rgb]{ .984,  .855,  .867}57.63 & \cellcolor[rgb]{ .984,  .961,  .973}64.41 & \cellcolor[rgb]{ .988,  .988,  1}66.10 & \cellcolor[rgb]{ .737,  .812,  .914}69.49 & \cellcolor[rgb]{ .973,  .435,  .443}30.51 \\
    ru    & \cellcolor[rgb]{ .863,  .902,  .957}67.80 & \cellcolor[rgb]{ .737,  .812,  .914}69.49 & \cellcolor[rgb]{ .482,  .631,  .824}72.88 & \cellcolor[rgb]{ .353,  .541,  .776}74.58 & \cellcolor[rgb]{ .984,  .961,  .973}64.41 & \cellcolor[rgb]{ .976,  .647,  .655}44.07 \\
    et    & \cellcolor[rgb]{ .98,  .827,  .839}55.93 & \cellcolor[rgb]{ .984,  .906,  .918}61.02 & \cellcolor[rgb]{ .863,  .902,  .957}67.80 & \cellcolor[rgb]{ .608,  .722,  .867}71.19 & \cellcolor[rgb]{ .737,  .812,  .914}69.49 & \cellcolor[rgb]{ .976,  .62,  .627}42.37 \\
    lt    & \cellcolor[rgb]{ .984,  .933,  .945}62.71 & \cellcolor[rgb]{ .98,  .725,  .733}49.15 & \cellcolor[rgb]{ .863,  .902,  .957}67.80 & \cellcolor[rgb]{ .988,  .988,  1}66.10 & \cellcolor[rgb]{ .988,  .988,  1}66.10 & \cellcolor[rgb]{ .973,  .412,  .42}28.81 \\\bottomrule
    \end{tabular}%
  \label{tab:webapps-bert}%
\end{table}%


% Table generated by Excel2LaTeX from sheet 'webapps_xlm-roberta-base_result'
\begin{table}[htbp]
  \centering
  \caption{Nodomu noteikšanas precizitāte uz \textit{Webapps} datukopas ar XLM-RoBERTa modeli, \%}
    \begin{tabular}{lrrrrrr}\toprule
    languages & 1a & 1b & 2a & 2b & 3a & 3b \\\midrule
    % languages & 1\_translated & 1\_untranslated & 2\_translated & 2\_untranslated & 3\_translated & 3\_untranslated \\\midrule
    en    &  --   & 42.37 &   --  & 69.49 &   --  & 64.41 \\
    lv    & 37.29 & 50.85 & 69.49 & 72.88 & 45.76 & 32.20 \\
    ru    & 35.59 & 44.07 & 71.19 & 67.80 & 42.37 & 38.98 \\
    et    & 32.20 & 44.07 & 74.58 & 62.71 & 50.85 & 37.29 \\
    lt    & 62.71 & 50.85 & 79.66 & 67.80 & 59.32 & 32.20 \\\bottomrule
    \end{tabular}%
  \label{tab:webapps-xml}%
\end{table}%


% Table generated by Excel2LaTeX from sheet 'webapps_xlm-roberta-base_result'
\begin{table}[htbp]
  \centering
  \caption{Nodomu noteikšanas precizitāte uz \textit{Webapps} datukopas ar XLM-RoBERTa modeli, \%}
    \begin{tabular}{lrrrrrr}\toprule
    languages & 1a & 1b & 2a & 2b & 3a & 3b \\\midrule
    % languages & 1\_translated & 1\_untranslated & 2\_translated & 2\_untranslated & 3\_translated & 3\_untranslated \\\midrule
    en    &   --    & \cellcolor[rgb]{ .98,  .725,  .733}42.37 &  --     & \cellcolor[rgb]{ .58,  .702,  .859}69.49 &  --     & \cellcolor[rgb]{ .69,  .78,  .898}64.41 \\
    lv    & \cellcolor[rgb]{ .976,  .569,  .576}37.29 & \cellcolor[rgb]{ .988,  .988,  1}50.85 & \cellcolor[rgb]{ .58,  .702,  .859}69.49 & \cellcolor[rgb]{ .506,  .647,  .831}72.88 & \cellcolor[rgb]{ .98,  .827,  .839}45.76 & \cellcolor[rgb]{ .973,  .412,  .42}32.20 \\
    ru    & \cellcolor[rgb]{ .973,  .514,  .522}35.59 & \cellcolor[rgb]{ .98,  .776,  .788}44.07 & \cellcolor[rgb]{ .541,  .675,  .843}71.19 & \cellcolor[rgb]{ .616,  .725,  .871}67.80 & \cellcolor[rgb]{ .98,  .725,  .733}42.37 & \cellcolor[rgb]{ .976,  .62,  .627}38.98 \\
    et    & \cellcolor[rgb]{ .973,  .412,  .42}32.20 & \cellcolor[rgb]{ .98,  .776,  .788}44.07 & \cellcolor[rgb]{ .467,  .624,  .82}74.58 & \cellcolor[rgb]{ .729,  .808,  .91}62.71 & \cellcolor[rgb]{ .988,  .988,  1}50.85 & \cellcolor[rgb]{ .976,  .569,  .576}37.29 \\
    lt    & \cellcolor[rgb]{ .729,  .808,  .91}62.71 & \cellcolor[rgb]{ .988,  .988,  1}50.85 & \cellcolor[rgb]{ .353,  .541,  .776}79.66 & \cellcolor[rgb]{ .616,  .725,  .871}67.80 & \cellcolor[rgb]{ .804,  .859,  .937}59.32 & \cellcolor[rgb]{ .973,  .412,  .42}32.20 \\\bottomrule
    \end{tabular}%
  \label{tab:webapps-xml}%
\end{table}%


\begin{figure}[h] 
   \centering
   \subcaptionbox{mBERT latviešu treniņdatu kopa}{\includegraphics[width=0.49\linewidth,trim={0 0.1cm 0 0},clip]{results-5/graphs/webapps_bert-base-multilingual-cased_lv-accuracy.png}}
   \subcaptionbox{mBERT mašīntulkoto latviešu treniņdatu kopa}{\includegraphics[width=0.49\linewidth,trim={0 0.1cm 0 0},clip]{results-5/graphs/webapps_bert-base-multilingual-cased_lv_en-accuracy.png}}
   \caption{caption} 
   \label{fig:webapps-bert}
\end{figure}


\begin{figure}[h] 
   \centering
   \subcaptionbox{mBERT apvienotā treniņdatu kopa}{\includegraphics[width=0.49\linewidth,trim={0 0.1cm 0 0},clip]{results-5/graphs/webapps_bert-base-multilingual-cased_all-accuracy.png}}
   \subcaptionbox{mBERT apvienotā mašīntulkoto treniņdatu kopa}{\includegraphics[width=0.49\linewidth,trim={0 0.1cm 0 0},clip]{results-5/graphs/webapps_bert-base-multilingual-cased_all_en-accuracy.png}}
   \caption{caption} 
   \label{fig:webapps-bert-all}
\end{figure}


\begin{figure}[h] 
   \centering
   \subcaptionbox{mBERT angļu treniņdatu kopa}{\includegraphics[width=0.49\linewidth,trim={0 0.1cm 0 0},clip]{results-5/graphs/webapps_bert-base-multilingual-cased_en-accuracy.png}}
   \subcaptionbox{XLM-RoBERTa angļu treniņdatu kopa}{\includegraphics[width=0.49\linewidth,trim={0 0.1cm 0 0},clip]{results-5/graphs/webapps_xlm-roberta-base_en-accuracy.png}}
   \caption{caption} 
   \label{fig:webapps-bert-xlm-en}
\end{figure}


\begin{figure}[h] 
   \centering
   \subcaptionbox{XLM-RoBERTa latviešu treniņdatu kopa}{\includegraphics[width=0.49\linewidth,trim={0 0.1cm 0 0},clip]{results-5/graphs/webapps_xlm-roberta-base_lv-accuracy.png}}
   \subcaptionbox{XLM-RoBERTa mašīntulkoto latviešu treniņdatu kopa}{\includegraphics[width=0.49\linewidth,trim={0 0.1cm 0 0},clip]{results-5/graphs/webapps_xlm-roberta-base_lv_en-accuracy.png}}
   \caption{caption} 
   \label{fig:webapps-xlm}
\end{figure}


\begin{figure}[h] 
   \centering
   \subcaptionbox{XLM-RoBERTa apvienotā treniņdatu kopa}{\includegraphics[width=0.49\linewidth,trim={0 0.1cm 0 0},clip]{results-5/graphs/webapps_xlm-roberta-base_all-accuracy.png}}
   \subcaptionbox{XLM-RoBERTa apvienotā mašīntulkoto treniņdatu kopa}{\includegraphics[width=0.49\linewidth,trim={0 0.1cm 0 0},clip]{results-5/graphs/webapps_xlm-roberta-base_all_en-accuracy.png}}
   \caption{caption} 
   \label{fig:webapps-xlm-all}
\end{figure}




\chapter*{Secinājumi}
\addcontentsline{toc}{chapter}{Secinājumi}

Testējot multilingual BERT un XLM-RoBERTa daudzvalodu jēdzientelpas lietotāju nodomu noteikšanā piecās dažādās valodās tika konstatēts, ka (modelim) bija visaugstākā precizitāte (valodās) valodās ar ()\%. (Modelis) arī darbojās labi ar kopējo precizitāti ()\%. Rezultāti liecina, ka daudzvalodu vārdu iegulšanas izmantošana var būt efektīva nodomu noteikšanai vairākās valodās, un modeļa izvēle var būtiski ietekmēt klasifikācijas uzdevuma precizitāti.

Modeļu precizitāte katrai valodai bija atšķirīga, ar zemāko precizitāti (valodā) valodā, un augstāko precizitāti (valodā) valodā, kas bija sagaidāms ņemot vērā datu kopas uz kurām tika apmācīti multilingual BERT un XLM-RoBERTa modeļi. 


Kopumā paredzams, ka nolūku klasifikācijas modeļu precizitāte būs augstāka oriģinālajiem ievades datiem latviešu, igauņu, krievu un lietuviešu valodā, salīdzinot ar to pašu ievades datu mašīnu, kas tulkota angļu valodā. Tas ir tāpēc, ka mašīntulkošana rada papildu trokšņus un kļūdas, kas var ietekmēt ievades datu kvalitāti un klasifikācijas modeļu veiktspēju. Turklāt mašīntulkošanas rezultātā dažkārt var tikt zaudētas svarīgas nianses un katrai valodai raksturīgā semantiskā informācija, kas var vēl vairāk pasliktināt ievades datu kvalitāti un apgrūtināt precīzu nolūku klasificēšanu.

Tomēr precīza modeļu veiktspēja var atšķirties atkarībā no vairākiem faktoriem, piemēram, mašīntulkošanas kvalitātes, ievades vaicājumu sarežģītības un izmantoto daudzvalodu vārdu iegulšanas specifiskajām īpašībām. Tāpēc ir svarīgi novērtēt modeļu veiktspēju gan ar oriģinālajiem, gan mašīntulkotajiem ievades datiem un salīdzināt rezultātus, lai labāk izprastu modeļu stiprās puses un ierobežojumus daudzvalodu nolūku klasifikācijas uzdevumiem.




%\chapter*{Pateicības}
%\addcontentsline{toc}{chapter}{Pateicības}
%Paldies Aleksandrai Elbakjanai par sci-hub.



\clearpage
\addcontentsline{toc}{chapter}{Izmantotā literatūra un avoti}
\printbibliography[title=Izmantotā literatūra un avoti]

\clearpage

\appendix
\chapter*{Pielikums}
\addcontentsline{toc}{chapter}{Pielikums}
% the \\ insures the section title is centered below the phrase: AppendixA
%\noindent {\large \textbf{Dalība konferencēs}}

\noindent {\large \textbf{Kods}}\\
Koda piemērs literatūras ievadā.
Krāsu dati "xkcd.json"
\url{https://github.com/dariusk/corpora/blob/master/data/colors/xkcd.json}.\\
Ideja \url{https://gist.github.com/aparrish/2f562e3737544cf29aaf1af30362f469}

\begin{python}
import numpy as np
import json

def hex_to_int(s):
    s = s.lstrip("#")
    return int(s[:2], 16), int(s[2:4], 16), int(s[4:6], 16)


def distance(coord1, coord2):
	"""Euclidean distance between two points
	"""
	return np.sqrt(np.sum(np.subtract(coord1, coord2)**2))


def subtractv(coord1, coord2):
	"""coord1 - coord2
	"""
	return np.subtract(coord1, coord2)


def addv(coord1, coord2):
	"""coord1 + coord2
	"""
	return np.sum([coord1, coord2], axis=0)


def closest(space, coord, n=10):
	"""as in https://gist.github.com/aparrish/2f562e3737544cf29aaf1af30362f469
	"""
	closest = []
	for key in sorted(space.keys(),
						key=lambda x: distance(coord, space[x]))[:n]:
		closest.append(key)
	return closest


color_data = json.loads(open("xkcd.json").read())

colors = dict()
for item in color_data['colors']:
    colors[item["color"]] = hex_to_int(item["hex"])

\end{python}

Darba rezultātu iegūšanai izmantotais kods pieejams arī: \url{https://github.com/chararchter/intent-detection}.

Datu korpusa mašīntulkošana angļu valodā.

\begin{python}

from typing import List

from transformers import pipeline

from model import read_file


def get_source_text(dataset_type: str, source_language: str, dataset_name: str) -> List[str]:
    """ Wrapper for get_data that provides file path.

    :param dataset_type: "test" or "train"
    :param source_language: "lv", "ru", "et", "lt"
    :param dataset_name: "chatbot", "askubuntu" or "webapps"
    :return: array of file contents for specified file
    """
    return read_file(f"NLU-datasets\{dataset_name}\{source_language}\{dataset_name}_{dataset_type}_q.txt")


def translate_to_file(dataset_type: str, source_language: str, dataset: List[str], dataset_name: str, model_name: str):
    """ Write the translated text to file.
    utf-8 encoding is specified in case the source text wasn't translated and still has the source language characters.

    :param dataset_type: "test" or "train"
    :param source_language: "lv", "ru", "et" or "lt"
    :param dataset: dataset of source_language and type e.g. "lv_train"
    :param dataset_name: "chatbot", "askubuntu" or "webapps"
    :param model_name: model name e.g. "opus-mt-tc-big-et-en"
    """
    pipe = pipeline("translation", model=model_name)
    for line in dataset:
        print(line)
        output = pipe(line)
        print(output)
        if "error" in output:
            print(output)
            write_to_file(source_language, dataset_type, dataset_name, "error, original line:" + line)
        else:
            write_to_file(source_language, dataset_type, dataset_name, output[0]["translation_text"])


def write_to_file(source_language: str, dataset_type: str, dataset_name: str, output: str):
    """ Write the translated text to file.
    utf-8 encoding is specified in case the source text wasn't translated and still has the source language characters.

    :param source_language: "lv", "ru", "et" or "lt"
    :param dataset_type: "test" or "train"
    :param dataset_name: "chatbot", "askubuntu" or "webapps"
    :param output: translated text or error and sentence in original language
    """
    with open(f"{dataset_name}_{source_language}_{dataset_type}.txt", "w", encoding="utf-8") as f:
        f.write(output + "\n")


def translate_to_english(dataset: List[str], model_name: str, source_language: str, dataset_type: str):
    pipe = pipeline("translation", model=model_name)
    translated_text = pipe(dataset)
    print(translated_text)
    write_to_file(source_language, dataset_type, translated_text)


model = dict()
model = {
    "lv": ".\opus-mt-tc-big-lv-en",
    "ru": ".\opus-mt-ru-en",
    "et": ".\opus-mt-tc-big-et-en",
    "lt": ".\opus-mt-tc-big-lt-en",
}

for language, model_name in model.items():
    for dataset_name in ["chatbot", "webapps", "askubuntu"]:
        for dataset_type in ["test", "train"]:
            dataset = get_source_text(dataset_type, language, dataset_name)
            translate_to_file(
                source_language=language, dataset_type=dataset_type, dataset_name=dataset_name, dataset=dataset,
                model_name=model_name
            )

\end{python}


Webapps datu kopas nepietiekami pārstāvēto anotāciju pārgrupēšana "Other" kategorijā.

\begin{python}
from collections import Counter

from model import get_source_text


def get_labels(dataset: str = "webapps") -> dict:
  """ Initialize data dictionary with train_labels and test_labels
  """
  data = dict()
  for dataset_type in ["test", "train"]:
    data.update(
      {f"{dataset_type}_labels": get_source_text(
      dataset_type=dataset_type, dataset=dataset, labels=True)}
    )
  return data


def clean_data(data: list) -> list:
  # Define the list of valid categories
  valid_categories = [
    'Find Alternative', 'Filter Spam', 'Sync Accounts', 'Delete Account'
  ]

  # Iterate through the list and check each element
  for i in range(len(data)):
    if data[i] not in valid_categories:
      data[i] = 'Other'
  return data


def count_categories(data: list):
  # Count the occurrences of each category
  category_counts = Counter(data)

  for category, count in category_counts.items():
    if count > 2:
      print(f'{category}: {count}')


labels = get_labels()
print(labels)

count_categories(labels["train_labels"])
print("######")
count_categories(labels["test_labels"])

# keepers: 
# 'Find Alternative', 'Filter Spam', 'Sync Accounts', 'Delete Account'

labels["train_labels"] = clean_data(labels["train_labels"])
labels["test_labels"] = clean_data(labels["test_labels"])

for dataset_type in ["test", "train"]:
  with open(f"webapps_{dataset_type}_ans.txt", "w", encoding="utf-8") as f:
    f.writelines(line + "\n" for line in labels[f"{dataset_type}_labels"])
\end{python}


Modeļa arhitektūra

\begin{python}
from typing import List

import matplotlib.pyplot as plt
import tensorflow as tf
from keras.layers import Dense, Conv1D, Dropout, GlobalMaxPooling1D, \
  MaxPooling1D
from keras.models import Sequential
from sklearn.model_selection import train_test_split
from transformers import BertTokenizer, TFBertModel, AutoTokenizer, TFAutoModel


def get_embeddings_tokenizer_model(model_name: str):
  model_name = f"./{model_name}"
  if "roberta" in model_name:
    return AutoTokenizer.from_pretrained(
      model_name), TFAutoModel.from_pretrained(model_name)
  else:
    return BertTokenizer.from_pretrained(
      model_name), TFBertModel.from_pretrained(model_name)


def read_file(path: str) -> List[str]:
  """ Read path and append each line without \n as an element to an array.
  Encoding is specified to correctly read files in Russian.
  Example output:
  ['FindConnection', 'FindConnection', ..., 'FindConnection']
  """
  with open(path, encoding='utf-8') as f:
    array = []
    for line in f:
      array.append(line.rstrip("\n"))
    return array


def get_source_text(
    dataset_type: str, dataset: str,
    source_language: str = None, labels: bool = False,
    machine_translated: bool = False
) -> List[str]:
  """ Wrapper for read_file that provides file path.
  Prompts in all languages are in the same order, therefore they use
  the same label files. So please be careful
  to use the correct argument for labels, as label=True returns labels
  regardless of specified source_language
  Usage examples:
  prompts: read_source_text("test", "et", False)
  labels: read_source_text("test")
  :param dataset_type: "test" or "train"
  :param dataset: "chatbot", "askubuntu" or "webapps"
  :param source_language: "lv", "ru", "et", "lt"
  :param labels: does the file being read contain labels
  :param machine_translated: has the data been machine translated to English?
  :return: array of file contents for specified file
  """
  if labels:
    return read_file(
      f"NLU-datasets\{dataset}\{dataset}_{dataset_type}_ans.txt"
    )
  elif machine_translated:
    return read_file(
      f"machine-translated-datasets\{dataset}_{source_language}_{dataset_type}.txt"
    )
  else:
    return read_file(
      f"NLU-datasets\{dataset}\{source_language}\{dataset}_{dataset_type}_q.txt"
    )


def get_dataset(datasets: dict, dataset: str = "chatbot") -> dict:
  """
  :param datasets: test/train key and languages as values
  :param dataset: "chatbot", "askubuntu" or "webapps"
  :return: dictionary with dataset type, language and optional
     labels and '_en' as keys and list of input data as values
  """
  results = dict()
  for key, value in datasets.items():
    results.update({f"{key}_labels": get_source_text(
      dataset_type=key, dataset=dataset, labels=True)})
    for lang in value:
      results.update({f"{key}_{lang}": get_source_text(
        dataset_type=key, dataset=dataset, source_language=lang)})
      if lang != "en":
        results.update({f"{key}_{lang}_en": get_source_text(
          dataset_type=key, dataset=dataset,
          source_language=lang, machine_translated=True)})
  return results


def split_train_data(x: list, y: list, validation_size: int = 0.2):
  """ Split training set in training and validation
  :param x: data
  :param y: labels
  :param validation_size: what fraction of data to allocate to training?
  :return:
  """
  return train_test_split(x, y, test_size=validation_size, stratify=y,
                          random_state=42)


def split_validation(datasets: dict, data: dict) -> dict:
  """ Split training dataset in training and validation
  :param datasets: dictionary with dataset type as key and list of languages as value
  :param data: dictionary with test/train, language and labels as key and data as values
  :return: updated data dictionary where each train key is split in train and validation
  """
  for key, value in datasets.items():
    if key == "train":
      for lang in value:
        data[f"{key}_{lang}"], \
          data[f"{key}_{lang}_validation"], \
          data[f"{key}_{lang}_labels"], \
          data[f"{key}_{lang}_labels_validation"] = split_train_data(
          data[f"{key}_{lang}"],
          data[f"{key}_labels"])
        if lang != "en":
          data[f"{key}_{lang}_en"], \
            data[f"{key}_{lang}_en_validation"], \
            data[f"{key}_{lang}_en_labels"], \
            data[f"{key}_{lang}_en_labels_validation"] = split_train_data(
            data[f"{key}_{lang}_en"],
            data[f"{key}_labels"])
  return data


def plot_performance(training_data, validation_data, broad_dataset: str,
                     dataset: str, x_label: str = 'accuracy'):
  plt.plot(training_data, label='training')
  plt.plot(validation_data, label='validation')
  ax = plt.gca()
  ax.set_xlabel('epochs')
  ax.set_ylabel(x_label)
  plt.title(f"{dataset} model {x_label}")
  plt.legend(loc="center")
  plt.savefig(f"graphs/{broad_dataset}_{dataset}-{x_label}.png")
  plt.show()


def create_model(sentence_length: int, num_classes: int = 2,
                 hidden_size: int = 768):
  model = Sequential()
  model.add(tf.keras.Input(shape=(sentence_length, hidden_size)))
  model.add(Dense(64, activation='relu'))
  model.add(Conv1D(128, kernel_size=3, activation='relu'))
  model.add(MaxPooling1D(pool_size=2))
  model.add(Conv1D(256, kernel_size=3, activation='relu'))
  model.add(MaxPooling1D(pool_size=2))
  model.add(Conv1D(512, kernel_size=3, activation='relu'))
  model.add(GlobalMaxPooling1D())
  model.add(Dropout(0.1))
  model.add(Dense(256, activation='relu'))
  model.add(Dense(num_classes, activation='softmax'))
  return model


def create_adam_optimizer(
    lr=0.001, beta_1=0.9, beta_2=0.999, weight_decay=0,
    epsilon=0, amsgrad=False, clipnorm=1.0):
  # sgd is worse than adam
  return tf.keras.optimizers.Adam(
    learning_rate=lr, beta_1=beta_1,
    beta_2=beta_2, epsilon=epsilon,
    amsgrad=amsgrad, weight_decay=weight_decay, clipnorm=clipnorm
  )


def get_classification_model(
    learning_rate: float, sentence_length: int,
    num_classes: int, clipnorm: float = 1.0
):
  optimizer = create_adam_optimizer(lr=learning_rate, clipnorm=clipnorm)
  classification_model = create_model(
    sentence_length=sentence_length, num_classes=num_classes
  )
  classification_model.compile(
    optimizer=optimizer,
    loss='categorical_crossentropy',
    metrics=['accuracy']
  )
  return classification_model


def training(data, lang: str, learning_rate: float, sentence_length: int,
             batch_size: int, epochs: int,
             model_name: str, broad_dataset: str, num_classes: int = 2):
  train_data = data[f"train_{lang}"]
  train_labels = data[f"train_{lang}_labels"]
  validation_data = data[f"train_{lang}_validation"]
  validation_labels = data[f"train_{lang}_labels_validation"]

  classification_model = get_classification_model(
    learning_rate, sentence_length, num_classes
  )

  history = classification_model.fit(
    train_data,
    y=train_labels,
    batch_size=batch_size,
    epochs=epochs,
    validation_data=(validation_data, validation_labels)
  )

  plot_performance(
    history.history['accuracy'],
    history.history['val_accuracy'],
    broad_dataset=broad_dataset,
    dataset=f"{model_name}_{lang}",
    x_label='accuracy'
  )

  plot_performance(
    history.history['loss'],
    history.history['val_loss'],
    broad_dataset=broad_dataset,
    dataset=f"{model_name}_{lang}",
    x_label='loss'
  )

  return classification_model


def test_classification_model(
    model, data: dict, lang: str,
    batch_size: int) -> float:
  test_data = data[f"test_{lang}"]
  test_labels = data["test_labels"]

  test_loss, test_accuracy = model.evaluate(
    test_data, test_labels,
    batch_size=batch_size)
  print('Test Loss: {:.2f}'.format(test_loss))
  print('Test Accuracy: {:.2f}'.format(test_accuracy))
  return test_accuracy

\end{python}



Eksperimentālo mērījumu iegūšana


\begin{python}
from typing import Iterable, Tuple

import pandas as pd
import tensorflow as tf
from keras.utils import to_categorical
from sklearn.preprocessing import LabelEncoder
from tensorflow.python.framework.ops import EagerTensor

from model import training, \
    test_classification_model, get_source_text, split_train_data, get_embeddings_tokenizer_model


class MyModel:
    def __init__(self, batch_size: int, learning_rate: float, epochs: int, sentence_length: int, model_name: str,
                 num_classes: int, dataset: str = "chatbot", languages=("en", "lv", "ru", "et", "lt")):
        self.batch_size = batch_size
        self.learning_rate = learning_rate
        self.epochs = epochs
        self.sentence_length = sentence_length
        self.model_name = model_name
        self.data = dict()
        self.datasets = dict()
        self.results = pd.DataFrame()
        self.num_classes = num_classes
        self.dataset = dataset
        self.hidden_size = 768
        # allows to run model one language at the time
        self.languages = [languages] if isinstance(languages, str) else languages
        self.non_eng_languages = list(set(self.languages) - {"en"})
        self.non_eng_languages = [language + "_en" for language in self.non_eng_languages]
        # explicit better than implicit, making sure the order of languages is consistent across board
        self.non_eng_languages = ['lv_en', 'ru_en', 'lt_en', 'et_en', 'lt_en']
        self.csv_file_name = f"{self.dataset}_{self.model_name}_results.csv"

        self.tokenizer, self.model = get_embeddings_tokenizer_model(self.model_name)

        self.init_dataset()
        self.init_data()
        self.init_results()

    def init_dataset(self):
        self.datasets = {
            "test": self.languages,
            "train": self.languages
        }

    def init_results(self):
        self.results['hyperparameters'] = [self.model_name, self.batch_size, self.sentence_length, self.learning_rate,
                                           self.epochs]
        self.results['languages'] = self.languages

    def init_data(self):
        self.get_dataset()

        self.labels_to_categorical()

        self.split_validation()

        self.convert_to_embeddings()

        self.merge_all_data('train_all', ['train_en', 'train_lv', 'train_ru', 'train_et', 'train_lt'])
        self.merge_all_data('train_all_labels',
                            ['train_en_labels', 'train_lv_labels', 'train_ru_labels', 'train_et_labels',
                             'train_lt_labels'])
        self.merge_all_data('train_all_validation',
                            ['train_en_validation', 'train_lv_validation', 'train_ru_validation', 'train_et_validation',
                             'train_lt_validation'])
        self.merge_all_data('train_all_labels_validation', ['train_en_labels_validation', 'train_lv_labels_validation',
                                                            'train_ru_labels_validation', 'train_et_labels_validation',
                                                            'train_lt_labels_validation'])

        self.merge_all_data('train_all_en', ['train_en', 'train_lv_en', 'train_ru_en', 'train_et_en', 'train_lt_en'])
        self.merge_all_data('train_all_en_labels',
                            ['train_en_labels', 'train_lv_en_labels', 'train_ru_en_labels', 'train_et_en_labels',
                             'train_lt_en_labels'])
        self.merge_all_data('train_all_en_validation',
                            ['train_en_validation', 'train_lv_en_validation', 'train_ru_en_validation',
                             'train_et_en_validation',
                             'train_lt_en_validation'])
        self.merge_all_data('train_all_en_labels_validation',
                            ['train_en_labels_validation', 'train_lv_en_labels_validation',
                             'train_ru_en_labels_validation',
                             'train_et_en_labels_validation', 'train_lt_en_labels_validation'])

        print(self.data)

    def is_translated(self, translated: bool = False) -> Tuple[list, list, str, str]:
        """ Return appropriate parameters based on whether the data has been machine translated to English
        :param translated: has the data been machine translated to English?
        :return: which list of languages to use, initialized result set and column name for results dataframe
        """
        if translated:
            return self.non_eng_languages, [None], "translated", "_en"
        else:
            return self.languages, [], "untranslated", ""

    def train_and_test_on_same_language(self, translated: bool = False):
        """ Each language has its own model, e.g., training on Latvian, testing on Latvian
        :param translated: has the data been machine translated to English?
        """
        temp_languages, temp_results, col_name, discard = self.is_translated(translated)
        for language in temp_languages:
            classification = training(self.data, language, self.learning_rate, self.sentence_length, self.batch_size,
                                      self.epochs, self.model_name, self.dataset, self.num_classes)
            temp_results.append(test_classification_model(classification, self.data, language, self.batch_size))
        self.results[f"1_{col_name}"] = temp_results
        self.results.to_csv(self.csv_file_name, index=False)

    def train_on_all_languages_test_on_one(self, translated: bool = False):
        """ One model trained on all language datasets, tested on each language separately
        e.g., training on all, testing on Latvian
        :param translated: has the data been machine translated to English?
        """
        temp_languages, temp_results, col_name, identifier = self.is_translated(translated)
        classification = training(self.data, f"all{identifier}", self.learning_rate, self.sentence_length,
                                  self.batch_size, self.epochs, self.model_name, self.dataset, self.num_classes)

        for language in temp_languages:
            temp_results.append(test_classification_model(classification, self.data, language, self.batch_size))
        self.results[f"2_{col_name}"] = temp_results
        self.results.to_csv(self.csv_file_name, index=False)

    def train_on_english_test_on_non_english(self, translated: bool = False):
        """ Trained on English only, tested on non-English
        e.g., training on English, testing on Latvian
        :param translated: has the data been machine translated to English?
        """
        temp_languages, temp_results, col_name, discard = self.is_translated(translated)
        classification = training(self.data, "en", self.learning_rate, self.sentence_length,
                                  self.batch_size, self.epochs, self.model_name, self.dataset, self.num_classes)

        for language in temp_languages:
            temp_results.append(test_classification_model(classification, self.data, language, self.batch_size))
        self.results[f"3_{col_name}"] = temp_results
        self.results.to_csv(self.csv_file_name, index=False)

    def get_dataset(self):
        """ Initialize data dictionary with values read from files
        """
        for key, value in self.datasets.items():
            self.data.update({f"{key}_labels": get_source_text(dataset_type=key, dataset=self.dataset, labels=True)})
            for lang in value:
                self.data.update(
                    {f"{key}_{lang}": get_source_text(dataset_type=key, dataset=self.dataset, source_language=lang)}
                )
                if lang != "en":
                    self.data.update({f"{key}_{lang}_en": get_source_text(
                        dataset_type=key,
                        dataset=self.dataset,
                        source_language=lang,
                        machine_translated=True
                    )})

    def labels_to_categorical(self):
        """ Convert string labels to categorical data
        """
        label_encoder = LabelEncoder()
        for key in self.data.keys():
            if "labels" in key:
                # Encode string labels to integer labels
                self.data[key] = label_encoder.fit_transform(self.data[key])
                # Convert integer labels to categorical data
                print(f"Num classes: {len(label_encoder.classes_)}")
                self.data[key] = to_categorical(self.data[key], num_classes=len(label_encoder.classes_))

    def split_validation(self):
        """ Split training dataset in training and validation
        """
        for key, value in self.datasets.items():
            if key == "train":
                for lang in value:
                    self.data[f"{key}_{lang}"], \
                        self.data[f"{key}_{lang}_validation"], \
                        self.data[f"{key}_{lang}_labels"], \
                        self.data[f"{key}_{lang}_labels_validation"] = split_train_data(self.data[f"{key}_{lang}"],
                                                                                        self.data[f"{key}_labels"])
                    if lang != "en":
                        self.data[f"{key}_{lang}_en"], \
                            self.data[f"{key}_{lang}_en_validation"], \
                            self.data[f"{key}_{lang}_en_labels"], \
                            self.data[f"{key}_{lang}_en_labels_validation"] = split_train_data(
                            self.data[f"{key}_{lang}_en"],
                            self.data[f"{key}_labels"])

    def convert_to_embeddings(self):
        """ Loop through the data dictionary and call get_word_embeddings on each key that isn't a label
        """
        for key, value in self.data.items():
            if "labels" not in key:
                self.data[key] = self.get_word_embeddings(value)

    def get_word_embeddings(self, vectorizable_strings: list) -> EagerTensor:
        """ Convert input to word embeddings
        """

        encoded_input = self.tokenizer(
            vectorizable_strings,
            padding='max_length',
            max_length=self.sentence_length,
            truncation=True,
            return_tensors='tf'
        )
        return self.model(encoded_input)["last_hidden_state"]

    def merge_all_data(self, new_key_name: str, keys_to_merge: Iterable):
        new_values = tf.concat([self.data[key] for key in keys_to_merge], axis=0)
        self.data.update({new_key_name: new_values})


# learning_rate=0.0001 is too small
if __name__ == "__main__":
    model = MyModel(batch_size=24, learning_rate=0.001, epochs=200, sentence_length=20, model_name="xlm-roberta-base")
    model.train_and_test_on_same_language(translated=True)
    model.train_and_test_on_same_language(translated=False)
    model.train_on_all_languages_test_on_one(translated=True)
    model.train_on_all_languages_test_on_one(translated=False)
    model.train_on_english_test_on_non_english(translated=True)
    model.train_on_english_test_on_non_english(translated=False)

    model = MyModel(batch_size=24, learning_rate=0.001, epochs=200, sentence_length=20,
                    model_name="bert-base-multilingual-cased", num_classes=2)
    model.train_and_test_on_same_language(translated=True)
    model.train_and_test_on_same_language(translated=False)
    model.train_on_all_languages_test_on_one(translated=True)
    model.train_on_all_languages_test_on_one(translated=False)
    model.train_on_english_test_on_non_english(translated=True)
    model.train_on_english_test_on_non_english(translated=False)


    model = MyModel(batch_size=16, learning_rate=0.0001, epochs=200, sentence_length=20,
                    model_name="bert-base-multilingual-cased", num_classes=5, dataset="askubuntu")
    model.train_and_test_on_same_language(translated=True)
    model.train_and_test_on_same_language(translated=False)
    model.train_on_all_languages_test_on_one(translated=True)
    model.train_on_all_languages_test_on_one(translated=False)
    model.train_on_english_test_on_non_english(translated=True)
    model.train_on_english_test_on_non_english(translated=False)
    
    model = MyModel(batch_size=16, learning_rate=0.0001, epochs=200, sentence_length=20,
                    model_name="xlm-roberta-base", num_classes=5, dataset="askubuntu")
    model.train_and_test_on_same_language(translated=True)
    model.train_and_test_on_same_language(translated=False)
    model.train_on_all_languages_test_on_one(translated=True)
    model.train_on_all_languages_test_on_one(translated=False)
    model.train_on_english_test_on_non_english(translated=True)
    model.train_on_english_test_on_non_english(translated=False)


    model = MyModel(batch_size=8, learning_rate=0.0001, epochs=200, sentence_length=20,
                    model_name="bert-base-multilingual-cased", num_classes=5, dataset="webapps")
    model.train_and_test_on_same_language(translated=True)
    model.train_and_test_on_same_language(translated=False)
    model.train_on_all_languages_test_on_one(translated=True)
    model.train_on_all_languages_test_on_one(translated=False)
    model.train_on_english_test_on_non_english(translated=True)
    model.train_on_english_test_on_non_english(translated=False)
    
    model = MyModel(batch_size=8, learning_rate=0.0001, epochs=200, sentence_length=20,
                    model_name="xlm-roberta-base", num_classes=5, dataset="webapps")
    model.train_and_test_on_same_language(translated=True)
    model.train_and_test_on_same_language(translated=False)
    model.train_on_all_languages_test_on_one(translated=True)
    model.train_on_all_languages_test_on_one(translated=False)
    model.train_on_english_test_on_non_english(translated=True)
    model.train_on_english_test_on_non_english(translated=False)
\end{python}


% Dokumentālā lapa
\makedoklapa

\end{document}
